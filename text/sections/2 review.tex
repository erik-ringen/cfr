\section{Foraging for children research in Human Behavioral Ecology}

\subsection{Becoming adult}
%early HBE little attention to children (focus on big hunting, !Kung)
%Draper and !Kung, not very active
%children foraging appear in other research on time allocation
%Kawabe 1983 and lifelong returns
%Children as dependent and unproductuve, waiting to become adults
Despite ethnographic or descriptive account of children foraging, quantitative research on these activities had a very slow development.

Among the first works focusing on children in small scale societies in Human Behavioural Ecology, \cite{lee_social_1976} strikes for how it depicts children as economically dependent from their families. Collecting time allocation data among the !Kung, a hunter gatherer population living in the Kalahari desert, she deduces that young individuals barely engage in productive activities independently from their parents. This is explained in part with the technologies available, and in part with the distribution of resources in the environment. The technologies !Kung employ to forage, according to Draper, are too simple and do not facilitate difficult tasks enough for children to engage with them. Also, low predictability in time and space of food resources contribute to excluding children from productive activities out of camp. 

In the following decade, \cite{nag_anthropological_1978} and \cite{munroe_childrens_1984} also measure time allocation of children. Focusing on in village and domestic settings, these papers offer a slightly different account of children work. In the societies they observe, children as young as three years old are reported spending at least a part of their time in productive/household activities. These include childcare and garden cultivation, but remarkably not foraging. Such an absence might be due to the fact that the studies are conducted in agricultural societies ('paesant societies' according to Nag et al.), but also to the method of data collection, that appear not to allow for observation in forests or other places where children would forage. 

Up to the late '80s, maybe the only paper reporting reporting quantitative accounts of children foraging is \cite{kawabe_development_1983}. The paper describes Gidra boys hunting and fishing in the lowlands of Papua New Guinea, but unfortunately it goes almost unnoticed in the contemporary literature. 

\subsection{Hadza vs !Kung: renewing the approach on children foraging returns}
%starting from the convention that children are useless, HBE is surprised to find out Hadza are very active
%explanatio of the difference between Hadza and !Kung (distance from food, predation risk, getting lost, weight balance for mothers; draper mentions simple technologies which do not allow children to perform hard tasks)
%family level optimization: ecological differences imply different optimal behaviors
%approach changes from 'why would children forage' to 'they must be optimizing something'

%citations needed

It is understandable, then, that researchers are surprised as they start to observe children collecting food resources in other societies (For example, \cite{blurton_jones_why_1997} set the stage of their paper against the backdrop of Tinbergen's four questions (\cite{tinbergen_aims_1963}) by exploring six different hypothesis of why indeed children would forage in a hunter gatherer population). %, titled "Why Do Hadza Children Forage?",

In particular, the question of why would children take up foraging is framed as a comparison between the !Kung and another African hunter gatherer population, the Hadza. Contrary to !Kung, Hadza children venture in the scrubland surrounding their villages in small mixed age groups to dig tubers, hunt small game or pick fruits  (\cite{hawkes_hadza_1995}, p. 694: "The Hadza are an exception to the conventional anthropological wisdom that the children of mobile hunter-gatherers do little to support themselves but instead rely completely on their parents for subsistence"). 

The contrast is sharpened by the fact that !Kung and Hadza are similar under various aspects. Considered remaining examples of the hunter gatherers populations which were occupying Eastern and Southern Africa before the Bantu agricultural expansions, both populations largely rely -or relied until recently- on wild products for food and shelter. They live in mobile camps following the availability of water and other resources, and have traditional diets comprising mainly tubers, nuts and fruits and wild meat (citation needed).

But the differences, in this case, count more than the similarities.
\cite{blurton_jones_foraging_1994, blurton_jones_differences_1994} describe the different hardships offered to Hadza and !Kung by their environment. Despite both populations live in arid, Savannah-like climate, Northern Tanzania, where the Hadza live, is a more favorable habitat. The landscape is more varied, reducing the chances for inexperienced children to get lost, and the risk offered by predators or other adversities is considered lower by Hadza parents. But the main factor seems to be the distance of resource patches from camp. Hadza children can collect up to half of their daily caloric needs by picking the fruits of baobabs, which often grow few hundred meters from the camp. In order to collect a comparable amount of calories, !Kung children should cover more than five kilometers to the closest Dobe nut grove. For the sake of comparison, no Hadza child goes that far from camp without adult supervision. Moreover, children do not follow their parent in the excursions to these groves, as adults report that the lack of water and shade along the way makes children more of a burden than help. 

In the same paper, Blurton Jones and colleagues suggest that for children to accompany the adults on these long trips is actually sub-optimal at the household level. Children appear to spend their time more productively by processing the food collected by their mothers, rather than collecting it themselves. More in detail, \cite{hawkes_hadza_1995} calculate the costs and benefits of different tasks children could accomplish, comparing data on various activities payoffs for the mother-offspring team. Beginning "with the simple hypothesis that children seek to maximize their mean rate of nutrient acquisition while foraging" (p. 694), they find that "Just as Hadza mothers earned higher team rates by taking their children to the distant berry patches than by foraging closer to camp, so !Kung mothers earn higher team rates by not taking their children to the mongongo groves, leaving them at home to crack nuts instead" (p. 697).

%(\cite{hawkes_hadza_1995}, p. 694: "We begin with the simple hypothesis that children seek to maximize their mean rate of nutrient acquisition while foraging."
%p. 697: "Just as Hadza mothers earned higher team rates by taking their children to the distant berry patches than by foraging closer to camp, so !Kung mothers earn higher team rates by not taking their children to the mongongo groves, leaving them at home to crack nuts instead. The difference lies in the processing requirements of the resources that provide the highest available team rates."
%p. 698: "The foraging patterns of Hadza children are determined by the age-specific return rates for local resources. Comparisons with the !Kung show that the processing requirements of these resources are also determinant."
%p. 699: "Two variables emerge as important in explaining the character of children's productive activity among mobile foragers. The first is the age-specific return rates for locally available resources that determine which alternatives give the highest team rates for women and children. The second is the character of the resources offering the highest team rates.")


From this in depth comparative analysis, emerges the main message that children optimal behavior depends on the environment and resources available. The work on juvenile foraging that follows tries to increase the amount and variability of data available, to understand how children and teenagers face cost-benefit trade offs in different conditions, as well as the evolutionary implications to foraging behaviors.


\subsection{A children's world}
%Children make sensible choices when foraging and have specific version of behavior (children face different survival and reproductive goals than adults e.g. they are not showing off, they're small and get what's appropriate for their size)
%Bird in Australia 

In particular, in the early 2000s, Rebecca Bliege Bird and her husband Douglas Bird published a series of papers on children foraging activities. For the first time, what children do was observed from their own point of view: "Paying attention to the differences from a child's perspective may help adults remember what being a child was all about." (\cite{bird_children_2002}, p292).

Working first on the island of Mer, in the Torres Strait Archipelago, and later on among the Mardu of Western Australia, the couple tried to understand the factors defining the trade offs children face when foraging. In both societies, children leave often the camp to hunt or fish, and are often collecting a large proportion of their daily caloric need ("Meriam children under age 15 each provided an average of 750 kcal/day", \cite{bliege_bird_children_1995}, p 16).

In \cite{bliege_bird_children_1995} and later more in detail in \cite{bird_ethnoarchaeology_2000} and \cite{bird_children_2002}, the authors observe how children and adult foraging behaviors differ on the intertidal reefs of Mer Island. They argue that the difference in the proportion of shell species collected is a function of adult vs. child body size, and that this difference is consistent with predictions from prey-choice models. 
%These models predict whether or not a prey should be pursued on encounter. They are used to understand decision making during foraging, because some preys carry a premium lower than the cost paid to capture/process/transport that prey, and thus are not worth collecting. Changes in diet composition would imply, according to this model, a change in the cost benefit ratio of certain species, or reduction in the abundance of others. In this case,
Specifically, children are smaller and walk at lower speed along the reef, which reduces their chance of encountering high value prey. It becomes optimal, then, to include lower value preys in the diet, and hence we observe the appearance of the small \textit{Trochus} and \textit{Strombus} in the children bounty, which are not collected by the adults.
%The \cite{bird_ethnoarchaeology_2000} paper is especially interesting, as it explicitly tests predictions from the prey choice model, predictions which are largely confirmed by real data. This has implications for the interpretation of archaeological data, as similar patterns could be originated as an effect of children foraging, as an alternative to other mechanisms such as a reduction in the abundance of certain species.
Similarly, in \cite{bird_mardu_2005}, Mardu children chose how to spend their foraging time in accordance with their size. Being slower than their parents, it is advantageous for them to hunt the smaller but more common lizards which live in rocky patches close to their village, instead of traveling to further patches where adults hunt. This way, they increase their total returns, by both reducing energy and time spent in traveling, and encountering preys at a higher rate 
%(as well as reducing the risk of dehydration or heat strokes and making time to care for their younger sibilings). 
Other factors defining what children forage might have nothing to do with their strength or size, but instead with what adults choose to prioritize. Social status or apparent reproductive value have different fitness consequences for parents and offspring, translating ultimately in different decisions in front of trade offs (\cite{bird_constraints_2002}). This has been used to explain big game hunting, which in some hunter gatherer societies has lower return rates than other less prestigious activities, but is still performed by individuals trying to increase their social capital or to impress a potential mate (citation needed \cite{}).   


%\cite{bliege_bird_children_1995}"We hope, through our continuing research on Mer, to understand w h y children would allocate time to foraging or other productive activities, and to understand what might account for the differences between children's and adults' foraging strategies." p. 3
%"Children of the Hadza in east Africa, for example, provide a daily average of 614 kcal towards their own subsistence (Blurton Jones et al 1989), while Ache children of Paraguay, under age 18, supply an average of 557 kcal/day, Machiguenga children in Peru produce 495 kcal/day, and, among the Piro of Peru, children produce 366 kcal/day (Kaplan nd). In comparison, in our non-randomly selected sample, Meriam children under age 15 each provided an average of 750 kcal/day, for either themselves or their families, in addition to an unknown amount gathered opportunistically in the village. Their return rates are based on lowest estimations of caloric content and are probably even higher than indicated in Table 1, but again the non-random nature of the sample suggests this figure is at the upper end for the Meriam children as a whole." p 16

%\cite{bird_children_2002} "Here we investigate whether differences in the prey choice of Meriam children and adults while reef-flat collecting are a result of attempts by children to learn adult strategies, or whether their efforts reflect differences in the constraints the children face while foraging on the reef." p 272
%"Paying attention to the differences from a child's perspective may help adults remember what being a child was all about." p292

\subsection{Kaplan embodied capital hypothesis and tests}
% Presented by Kaplan in some early papers and then more extensively in the 2000 paper
%     Human success is due to the exploitation of a complex foraging niche
%     Human Life history is an adaptation to it
%     Childhood because we need to get good at exploiting that niche (learning, growing, skills)
% This allowed to make predictions, that would be tested with data on children foraging
% In early 2000 several researchers collected data on children foraging to test Kaplan’ hypothesis
% Methods and conclusions vary, but they observed that children forage and that they get better with age
% But not homogeneously, there are differences between sexes and resources
% Especially hunting seems to peak late -walker, koster, discuss skill intensity of foraging, maybe gender differences

\cite{bird_children_2002} "Because a number of phenotypic traits are correlated with age, including size, experience, digestive morphology, and nutritional requirements, one cannot use simple age differences in behavior to support the hypothesis that a long period of juvenility is a response to the complexity of adult behavior."p292


\subsection{Cooperative breeding, pacing of reproduction and self sufficiency}
%Kramer, Crittenden, prob Kaplan and children self provision, cost of children in time and inter birth interval

\subsection{Learning and knowledge}
%Traditional ecological knowledge, play, social vs individual learning
% A Theory of Fertility and Parental Investment in Traditional and Modern Human Societies, Kaplan 1996

\subsection{Other emerging themes}
% On the side and especially in the last decade, other areas of research emerged that took an interest in children activities and foraging, although not all of them actually report foraging returns (many consider time allocation, which is no good for our metaanalysis, but interesting for theory)
% Many studies demonstrate that children forage consistently and contribute to a substantial amount of their caloric needs. This has implications for the understanding of different phases of life history, including the length of the pre reproductive period and inter birth interval. 
% Some studies started to have an interest in children foraging in other environment, such as urban
% Or in a context of environmental and social change, as Draper did first in 1988- Pollom 2020 on effects of ecological change on subsistance have to consider children
% Or finally there’s more attention on how children learn, which is tangential to the theme of foraging, trying to explain variation



Early studies in the field of Human Behavioral Ecology (HBE) paid little attention to children. The focus on hunting, especially big fauna, as main form of foraging and the high relevance in the HBE literature of a population in which children exhibit a particularly low levels of activity, the !Kung, contributed to this low level of interest. 

One of the earliest studies focusing on children activities in the !Kung \cite{lee_social_1976} describes the various factors limiting the possible contributions of children to food production. The dangerous environment, a simple bur somewhat unhelpful technology, the unpredictability of resources, all result in 'the absence of children work'.
