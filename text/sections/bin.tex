ABSTRACT


%meta-analysis we aim to summarize previous research both qualitatively and quantitatively. Starting from a cross culturl analysis published by \cite{koster_life_2019}, we developed a model to analyze published data on children foraging returns. 

%Children foraging activities interest different areas of the study of human behavioral ecology. The hunting and gathering of children and adolescents has a role in household economy and division of labor studies, cooperative breeding in humans and in investigating the evolution of life history traits. It is surprising, then, that data is scarce and a unifying framework is missing. In this paper, we compare the available data on children foraging returns in the current literature and argue that: - children forage extensively in several societies, - differences in productivity can be explained by ecological contingencies, - the improvements with age vary across culture and task. We also address the context in which research on children foraging has been conducted and call for an increase in data collection as well as for sharing protocols and methods. A concerted effort to create datasets comparable across cultures would allow to address important questions of human evolution.

INTRODUCTION
Foraging and diet of children - quotes
%\cite{mcgarry_children_2009}However, 62% of all the children interviewed (both non-school and school-going) supplemented their diets with wild foods. Dietary diversity showed a 13% increase when wild-food supplementation occurred. 
%\cite{fouts_who_2009} "Child feeding involvement by juvenile relatives (e.g., siblings and cousins) was not predicted by family transition, birth order, or physical ecology. However, it is striking to note that juvenile relatives provided food as much as adult female relatives, and nearly as much as elderly female relatives. Since birth order did not predict levels of child feeding by juvenile relatives, one could assume that even if a child is first born, with no older siblings, they are still receiving care from cousins. This is not surprising since the Aka live in large extended family groups and juveniles typically make up at least half of the population. However, the lack of birth order effect is somewhat contradictory to evolutionary hypotheses (see Turke, 1988) that suggest siblings in particular will provide care due to their close genetic relatedness compared to other alloparents with more distant genetic relatedness (e.g., cousins)." 
%Borgerson, Cortni, et al. "An evaluation of the interactions among household economies, human health, and wildlife hunting in the Lac Alaotra wetland complex of Madagascar." Madagascar Conservation & Development 13.1 (2018): 25-33. for indication that supplementing diet could reduce usage of wild foods

Foraging and knowledge
%For example, \cite{chpeniuk} observed that individuals who foraged a higher breath of goods have a better sense of biodiversity as adults.
%studies on how children interpret and conceptualize the natural environment often contrast data from groups where children have different levels of reliance on wild foods.
%cite paper on hunting in the us, and have a look at katja leibal and annie wertz work

Expansion of introduction
%effect of changing environments: if children foraging is important, and an environmental change makes it impossible, is there an effect on health? 
%Also, schooling: if children go to school and do not hunt, for example, how is this supplemented in their diet? Schools provide meals, but sometimes cheap carbs (Tanzania liquid porridge)

%info for archaeology and other fields

Other introduction drafts

%BIN

%Trends of  foraging return increases with age relative to average adult value have been modeled taking into account variation in resource collected, society of provenance and method of data collection. 
%Foraging requires several skills and imposes complex trade offs, that inevitably change as children grow and their bodies change. 

%foraging important for HBE and developmental trajectories interesting
%Foraging activities have been a central theme in Human Behavioral Ecology since its conception (\cite{laland_sense_2011}, chapter 4). This is due, in one hand, to the assumption that foraging returns are associated with reproductive fitness, which makes them relevant for evolutionary studies ("Consider a situation in which there is a direct relationship between time spent hunting, the amount of game acquired (measured in total weight or its equivalent in calories or nutrients), and the relative reproductive success of the hunter." in \cite{hawkes_how_1985}). And, in the other hand, the fact that returns can be clearly quantified put them at the center of several models producing testable predictions, as in optimal foraging theory (for a review, see \cite{smith_anthropological_1983}). 

%Later, researchers started focusing on the development of foraging skills, bringing attention to age specific returns. For example, \cite{koster_life_2019} developed a model that traces age trajectories of individual hunting returns in 40 study sites. 

%At the same time, the work of young individuals in subsistence societies became apparent through quantitative and qualitative analyses.
 
 
%why is the analysis of children behavioral returns important (evolution of life history, learning and mechanisms limiting returns, children welfare -e.g. nutritional status, social role and schooling-, impact of ecological and social change, all need info on children work, role in the family and also what and how much they forage.) 
%Understanding how children and adolescents interact with their environment to extract food resources has implications in various areas of the study of human behavior. 
%For example, nutritional status and children welfare are directly affected by their foraging activities: is the food they procure an important contribution to their caloric intake? Are the socializing opportunities that happen during this activity important for the development of individuals? The effects of social innovations, such as the increase in schooling time, or the ecological changes can have important impacts on children nutritional status. 
%Also, Being a proficient adult in any human society requires a vast array of skills, that are necessarily learnt over time. How returns change along the life of individuals can help clarify how this process of skill acquisition works.

%in order to do this, we need to better understand how children improve their foraging returns, and what contributes to this change, and also understand if there are general trends 


REVIEW
\subsection{Hadza vs !Kung: renewing the approach on children foraging returns}
%(\cite{hawkes_hadza_1995}, p. 694: "We begin with the simple hypothesis that children seek to maximize their mean rate of nutrient acquisition while foraging."
%p. 697: "Just as Hadza mothers earned higher team rates by taking their children to the distant berry patches than by foraging closer to camp, so !Kung mothers earn higher team rates by not taking their children to the mongongo groves, leaving them at home to crack nuts instead. The difference lies in the processing requirements of the resources that provide the highest available team rates."
%p. 698: "The foraging patterns of Hadza children are determined by the age-specific return rates for local resources. Comparisons with the !Kung show that the processing requirements of these resources are also determinant."
%p. 699: "Two variables emerge as important in explaining the character of children's productive activity among mobile foragers. The first is the age-specific return rates for locally available resources that determine which alternatives give the highest team rates for women and children. The second is the character of the resources offering the highest team rates.")

\subsection{A children's world}
%\cite{bliege_bird_children_1995}"We hope, through our continuing research on Mer, to understand w h y children would allocate time to foraging or other productive activities, and to understand what might account for the differences between children's and adults' foraging strategies." p. 3
%"Children of the Hadza in east Africa, for example, provide a daily average of 614 kcal towards their own subsistence (Blurton Jones et al 1989), while Ache children of Paraguay, under age 18, supply an average of 557 kcal/day, Machiguenga children in Peru produce 495 kcal/day, and, among the Piro of Peru, children produce 366 kcal/day (Kaplan nd). In comparison, in our non-randomly selected sample, Meriam children under age 15 each provided an average of 750 kcal/day, for either themselves or their families, in addition to an unknown amount gathered opportunistically in the village. Their return rates are based on lowest estimations of caloric content and are probably even higher than indicated in Table 1, but again the non-random nature of the sample suggests this figure is at the upper end for the Meriam children as a whole." p 16

%\cite{bird_children_2002} "Here we investigate whether differences in the prey choice of Meriam children and adults while reef-flat collecting are a result of attempts by children to learn adult strategies, or whether their efforts reflect differences in the constraints the children face while foraging on the reef." p 272
%"Paying attention to the differences from a child's perspective may help adults remember what being a child was all about." p292



\subsection{Evolution of childhood}
% Presented by Kaplan in some early papers and then more extensively in the 2000 paper
%     Human success is due to the exploitation of a complex foraging niche
%     Human Life history is an adaptation to it
%     Childhood because we need to get good at exploiting that niche (learning, growing, skills)
% This allowed to make predictions, that would be tested with data on children foraging
% In early 2000 several researchers collected data on children foraging to test Kaplan’ hypothesis
% Methods and conclusions vary, but they observed that children forage and that they get better with age
% But not homogeneously, there are differences between sexes and resources
% Especially hunting seems to peak late -walker, koster, discuss skill intensity of foraging, maybe gender differences

%\cite{bird_children_2002} "Because a number of phenotypic traits are correlated with age, including size, experience, digestive morphology, and nutritional requirements, one cannot use simple age differences in behavior to support the hypothesis that a long period of juvenility is a response to the complexity of adult behavior."p292

\subsection{Cooperative breeding, pacing of reproduction and self sufficiency}
%Maybe not relevant here, not many papers with return data in this section, but discussed in the intro and definitely an area where where expansion would be ideal...

%\cite{crittenden_juvenile_2013} Juveniles are both dependents and providers. This “two-fold” nature of human juvenility is an often overlooked, yet critical, dimension of human life history and the evolution of cooperative breeding (Kramer, 2011). Themajority of life history models evaluating juvenile contributionsmeasure net economic value, yet young foragers are concurrently collecting, exchanging, and receiving resources before they become a net producer (Kramer, 2011). Hadza juvenile foragers are no exception; although not collecting the entirety of their daily caloric requirements, they aremaking substantial contributions byway of self-provisioning. The caloric contributions provided by juveniles underwrite the cost of their care and may contribute to a mother's ability to successfully raise multiple dependents, thus supporting the notion of humans as cooperative breeders.

%\cite{kramer_maya_2005}
%The timing of the transition from net consumption to net production shapes the age range of children dependent on others ends
%If the span of time between when children become economically independent and when they are no longer available as keepers is variable across cultures, what conditions that variation is of interest because of its effect on the competing demands of parents having to support the disparate needs of both younger and older children and, ultimately, on the number of children parents can successfully raise.
%Given these concerns, neither money nor calories adequately reflect work effort in the Maya case. Consequently, time was selected as the most suitable currency for measuring an individual’s consumption and production.

%\cite{kramer_variation_2002}
%Maya children produce more than they consume by their early to mid teens but remain in their natal households for a number of years before leaving home and beginning families of their own. The Maya results contrast markedly with those from several groups of hunter-gatherers and horticulturists for whom we have similar data. Even in the Maya case, where children are self-sufficient at a relatively young age, parents are unable to support their children without help from others. The production surplus of older children appears to help underwrite the cost of large Maya families and subsidize their parents' continued reproduction.
%The timing of children's economic independence influences the parents' time and resource budgets in two important ways. First, in most animal species, juveniles become self-sufficient before or at sexual maturity, but among humans this may or may not be the case. When older children are able to support themselves, parents have a greater available budget to care for younger children if they so choose. Second, the production surplus of older children can be transferred to help subsidize dependents, allowing parents to raise more children than they might otherwise be able to do.


\subsection{Diet}
%\cite{lee_childrens_2009} does not find associations between child foraging and nutritional benefits in a Mexican shanty town, whereas \cite{mcgarry_children_2009} observe an increase in diet quality and variability thanks to foraged food in rural South Africa, especially in poorer households. Changing dietary patterns in hunter gatherers populations have an effect on the foraging habits of juveniles.

%The foraging activities of children have impacts on their diet. Outside of hunter gatherer populations, children have been observed foraging in marginalized urban (\cite{lee_childrens_2009}) and rural environments (\cite{mcgarry_children_2009}). Despite the growing body of evidence indicating that wild foods are beneficial for diet quality (add citations), the studies focusing on the impact of foraging activities on health status of children in precariours nutritional status are surprisingly few. 

More from review

% On the side and especially in the last decade, other areas of research emerged that took an interest in children activities and foraging, although not all of them actually report foraging returns (many consider time allocation, which is no good for our meta-analysis, but interesting for theory)
% Many studies demonstrate that children forage consistently and contribute to a substantial amount of their caloric needs. This has implications for the understanding of different phases of life history, including the length of the pre reproductive period and inter birth interval. 
% Some studies started to have an interest in children foraging in other environment, such as urban
% Or in a context of environmental and social change, as Draper did first in 1988- Pollom 2020 on effects of ecological change on subsistence have to consider children
% Or finally there’s more attention on how children learn, which is tangential to the theme of foraging, trying to explain variation
%patterns of food sharing emerge when
%learning


%Early studies in the field of Human Behavioral Ecology (HBE) paid little attention to children. The focus on hunting, especially big fauna, as main form of foraging and the high relevance in the HBE literature of a population in which children exhibit a particularly low levels of activity, the !Kung, contributed to this low level of interest. 

%One of the earliest studies focusing on children activities in the !Kung \cite{lee_social_1976} describes the various factors limiting the possible contributions of children to food production. The dangerous environment, a simple bur somewhat unhelpful technology, the unpredictability of resources, all result in 'the absence of children work'.
