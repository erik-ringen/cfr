\section{Foraging for children research: a review}

Despite descriptive accounts of children foraging behavior have been presented in many ethnographic studies, quantitative research on these activities had a slow development.

The first studies to quantify the returns children obtain from foraging were promoted by the observation of an interesting case-study. In two African hunter-gatherer populations, the !Kung and the Hadza, children exhibit radically different engagement in foraging, despite many similarities in lifestyle and Savannah-like environment. While the !Kung children spend most of their time in the village and collect almost none of the food they eat (\cite{lee_social_1976}, Hadza children are very active, move freely in and around the camp and contribute to their own maintenance (\cite{blurton_jones_differences_1994, crittenden_juvenile_2013}). Many factors seem to account for such a difference (\cite{blurton_jones_modelling_1989, blurton_jones_differences_1994, blurton_jones_foraging_1994, blurton_jones_why_1997}), mainly linked to the environment, such as availability of water, presence of predators and, most importantly, distance and type of resources. Compared to !Kung, Hadza children are able to forage autonomously, as a result of a relatively less hostile environment, with more resource patches close to the camp and less chances of getting lost or being attacked by predators. 
Moreover, among the Hadza, children routinely follow adults in foraging expeditions, which !Kung children never do. According to \cite{hawkes_hadza_1995}, both populations display an adaptive behavior: the total returns for the !Kung mother-child dyad are maximized when children process, at the village, the nuts their mothers collected far from it. On the contrary, Hadza mothers routinely bring their offspring along during foraging excursions to certain types of resource patches, such as berries, maximizing total returns according to age-specific yield rates for each resource.

That children and adults cooperate to maximize foraging returns is just one aspect of the fact that children strive to adopt optimal behavior. In the literature following the first studies cited above, children are not anymore just individuals waiting to become adult, but have specific goals and are full-fledged members of households. This emerges both from a series of papers that explored the specific trade-offs and constraints children face when foraging, and from a body of research focused on the role of children in the household, specifically in activities linked to cooperative breeding.  

To the first group of these papers belong the work of Bliege-Bird and Bird (\cite{bliege_bird_children_1995, bird_ethnoarchaeology_2000, bird_constraints_2002, bird_children_2002, bird_mardu_2005}). Working first with the people living on the Island of Mer, in the Torres strait, and later with the Mardu from Western Australia, they observed how children foraging give priorities to age specific set of preys. Interpreting this result in the light of diet breadth models (cite!), children target the appropriate species given their size, strength and resistance. They favor, for example, more common shells and lizards given they face higher cost of movement. 
Similarly, \cite{tucker_growing_2005} observes how Mikea children in Madagascar exploit more thoroughly the patches of tubers and focus on species that are easier to extract, despite they argue that children are not striving for efficiency. 

The theme of cooperative breeding in humans has been widely explored by \cite{kramer_variation_2002, kramer_maya_2005, kramer_childrens_2005, kramer_does_2009}, who demonstrated not only that children help to raise their siblings thorough childcare, but also that their contribution to the economy of their household is an important factor in pacing the reproduction of their mothers. This is linked to the emergence of two typically human features, not shared by the other great apes: short Inter Birth Interval and multiple dependent offspring.

Aside from implications of cooperative breeding, another aspect of human life history that is deeply intertwined with the evolution of childhood is the length and features of the pre-reproductive period itself. 
According to the Embodied Capital Hypothesis (ECH), a prolonged period before reproduction is needed to acquire some traits that allow to exploit the complex foraging niche humans live in. So children and adolescent would postpone reproduction in order to acquire strength or, more importantly, learn features of the environment what will allow a better extraction of resources and hence a higher fitness in the future. The ECH has been proposed in multiple papers by Kaplan and colleagues (\cite{kaplan_theory_2000, kaplan_theory_1996, kaplan_evolution_1997, kaplan_embodied_2001, kaplan_emergence_2002, kaplan_embodied_2003, kaplan_neural_2003, kaplan_life_2006, kaplan_evolution_2007}) and, later, several other authors took up the testing of such hypothesis. According to the ECH, the increase in foraging returns with age should follow the increase in the other traits which acquisition prolongs the pre-reproductive period.

The following papers try to test this association. 
\cite{bird_children_2002} collected naturalistic return data from children collecting seashells on the island of Mer, and also, in \cite{bird_constraints_2002}, from young Mardu hunting lizards in the Western Australian desert. \cite{bock_learning_2002} observed and measured both food collection and processing in a multi-ethnic community in the Okavango Delta of northwestern Botswana. And, finally, \cite{blurton_jones_selection_2002} organized competitions to measure the development of a combination of foraging-relevant techniques among the Hadza in Tanzania. 
Also in 2002, \cite{walker_age-dependency_2002} focused on hunting returns among the Ache of Paraguay with a combination of naturalistic observations, contests and recording of activities on a diary by informants. 
In 2005, \cite{bock_what_2005} published again on a variety of foraging activities carried on in the Okavango Delta, and \cite{tucker_growing_2005} contributed with data on tuber digging among the Mikea hunter-gatherers of Madagascar. 
\cite{gurven_how_2006}, combining interviews and naturalistic observations, reported hunting returns among the young and adult Tsimane horticulturalists of Bolivia and, most recently, \cite{crittenden_juvenile_2013} described children foraging for various products among the Hadza with naturalistic observations of returns.
Mainly, these papers agree that the way returns change with age depends on the specific constraints of the activity taken into consideration. Different types of embodied capital, such as growth or skill based, peak at different times for different resources according to the complexity of the task. Hunting, for example, appears to be a particularly skill intensive activity, with multiple combining elements each depending on different experience-based forms of capital (finding prey, tracking, archery skills), but also requires some strength and endurance. The data suggest that hunting might be one of the foraging activities for which returns peak later in life. 

Hunting is treated exhaustively by \cite{koster_life_2020}, with a large cross cultural database of juvenile and adult returns for more than 1,800 individuals from 40 societies, spanning several years. According to the study, age at peak hunting success is in general quite high, with an average between 30 and 35 years of age, but it also varies considerably across societies. Environmental variability in the nature as well as the frequency of encounter with prey might be at the root of the differences, highlighting how foraging success depends on the specific condition of the task. 

\cite{koster_life_2020} reinforces the idea that greater breath in the observed foraging data, in terms of environmental condition, but also of type of activity, is necessary to draw general conclusions on the evolution and development of human foraging and its implications for the evolution of human life history. 

This overview of the literature outlines the existing data, which come from a variety of populations and refer to different foraging activities, but also highlights the lack of a general perspective on the hunting and gathering conducted by pre-reproductive individuals. in here, we first offer a comparative view of children and adolescent foraging activities through a careful meta-analysis. Secondly, by pointing out the factors introducing variability in the analyzed data, we hope to provide a benchmark of comparison for future studies on the subject, which can help delve deeper in the many questions that involve the evolution of childhood. 