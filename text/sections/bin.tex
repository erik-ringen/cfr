ABSTRACT


\section*{Abstract}
Children forage in a vast array of societies, spanning from traditional hunter gatherer groups, to Mexican shanty towns. Researchers have repeatedly measured this behavior, mainly trying to explain the delayed age at first reproduction in our species with a need-to-learn hypothesis. But data on children foraging success is sparse and difficult to compare, and inferences on the evolution of early life history traits hard to draw. With this paper we aim at summarizing available return data in a single coherent meta-analytical framework and describe the curve of increases in foraging return with age. Moreover, the multilevel Bayesian model we employ allows to estimate the latent skill underlying foraging returns. We also consider different resources, starting from the assumption that more difficult resources should have a different curve of acquisition. We find that foraging skills increase with age, but with diminishing marginal returns. Foraging return rates differ between resources, with difficult-to-extract resources, such as tubers and game, showing an increase in production later in life than easy-to-acquire products, such as fruits. We conclude by indicating future directions for research aiming at understanding the evolution of childhood by looking at foraging return data.  


%complete the abstract with results and conclusions







% talk about need to distinguish drivers of complexity, tubers vs game--knowledge v strength



%In support of ECT, we also find variation in return rates between resources. 
%The human niche relies on foods with different levels of complexity, from fruits that simply need to be picked from trees, to mobile preys that require multiple individuals to coordinate in group hunting \citep{schuppli_life_2016}.
%The speed at which returns increase with age varies between these types of resources: while hunters may take up to 40 years to reach a peak in productivity \citep{koster_life_2020}, fruit picking can be mastered relatively easily by very young individuals. While previous studies failed to account for these differences, our results highlight the contrast from low/medium-complexity tasks, such as gathering of shellfish and fruits, and high-complexity foraging like extracting tubers from hard soil or hunting. 
%But \cite{schuppli_life_2016} describe complexity along another different axis. Some aspects of complexity involve motor skills, so that grazing is easier than picking fruits and ingesting them, which in turn is easier than extracting underground storage organs. But they also describe increasing complexity in the dimension of knowledge, so that knowing where and when to pick a certain ripe fruit requires knowledge of the environment. 
%Our results seem to indicate that fruits collection rates can increase incredibly fast in young humans, somewhat in contradiction with the idea that knowledge is one of the main "embodied capitals" for the acquisition of which humans delay reproduction. 
% But children do not live in a social vacuum, and they have access to direct information when foraging in groups with older, more knowledgeable individuals. Once a foraging group reaches a patch of ripe berries, the motor complexity aspect of foraging fruits is less of a limitation. On the contrary, USO can be incredibly hard to extract in the absence of well developed motor skills, which cannot be transferred between individuals. 

% Framing children's foraging behavior in the social settings they live in is hence important to understand the actual value of their output. Kids participate in a complex network of individuals that divide their labor in order to maximize the total output at the family level. San children stay home pounding nuts, while their mothers forage far from camp, because that is the most profitable task they can be allocated to cite(). In the absence of older members of the family, children adaptively modify their behavior, substituting their fathers in certain tasks, which they would not otherwise perform. The foraging activities should hence be interpreted in light of an age graded division of labor. 

% Other aspects of foraging might push children to pursue one resource rather than another. Children in different settings may be differently motivated (by hunger, play, prestige, etc.) to pursue different types of resources at different ages. Processing time is another important aspect of choosing what and when to collect, and not always the person who collects the food is the same who processes it, making it difficult to interpret foraging returns at the individual level.

% Given the large number of aspects influencing individual level returns, the data currently available are not sufficient to understand if the duration of the pre-reproductive period is functional in humans to the acquisition of an "embodied capital". The Direct Acyclic Graph we present in figure \ref{fig:DAG} is not sufficient to describe the whole complexity of factors linked to foraging, but it shows the few elements that we can compare cross-culturally with reasonable certainty. 

We recommend that future studies clearly justify their view of resource complexity according to (1) abundance & distribution of the resource, (2) strength needed, (3) tool use skill, (4) embodied skill (e.g. stalking, walking quietly), (5) cognitive skill (e.g. tracking), and (6) group coordination


schuppli and resource complexity. More complex resources require more time. BUt division between k complex and motor complex places fruits in K complex and USO in motor complex. Humans become fast at K complex fast, and slow in motor complex. Probably has to do with group cooperation, where individuals do not go foraging alone but benefit from k of other individuals. On the contrary motor cannot be transferred between individuals and hence only achieving sufficient skill level an individual can have beter results
Another aspect that we try to tackle/often overlooked in past research on children foraging return is the variation given by different resources. 

Humans have a very varied diet that spans from easily accessible berries, to underground storage organs, to mobile preys that require multiple individuals to coordinate in group hunting. The speed at which returns increase with age varies greatly between these types of resources and cannot be easily compared. Hunters may take up to 40 years to reach a peak in productivity \citep{koster_life_2020}, while fruit picking can be mastered relatively easily by very young individuals. While previous studies failed to account for these differences, our results highlight the contrast from low-complexity tasks, such as gathering of shellfish and fruits, and high-complexity foraging like extracting tubers from hard soil or hunting. The shape of foraging returns is markedly different between these two groups, offering support to the idea that more complex resources require longer periods of knowledge or skill acquisition to be exploited. 

Hard-to-get resources can required one or more aspects of "embodied capital" to be acquired, such as strength or tracking skills, and children probably pursue resources appropriate for their age, favoring easier foods while they're younger. 
%But they also engage in simulations of hunting and other adult tasks through play well before they actually start participating into these activities. 


Sharing within and between human families implies that no child, nor adult, is expected to produce the full amount of necessary calories on a specific day, but the timing and amount of food produced at each age has implications on when individuals would be able to start reproducing and how much/how successfully.

Framing children's foraging behavior in the social settings they live in is important to understand the actual value of their output. Kids participate in a complex network of individuals that divide their labor in order to maximize the total output at the family level. San children stay home pounding nuts, while their mothers forage far from camp, because that is the most profitable task they can be allocated to cite(). In the absence of older members of the family, children adaptively modify their behavior, substituting their fathers in certain tasks, which they would not otherwise perform. The foraging activities should hence be interpreted in light of an age graded division of labor. 

%This missing piece made it difficult, for past studies on children foraging return, to draw general conclusions on how much the need to acquire "embodied capital" has been important in shaping early human life history. We hence encourage future research to include family.
%Rather, we aim to provide summary of the available data and offer indications on how to address this issue more appropriately in the future. 


We recommend that future studies clearly justify their view of resource complexity according to (1) abundance & distribution of the resource, (2) strength needed, (3) tool use skill, (4) embodied skill (e.g. stalking, walking quietly), (5) cognitive skill (e.g. tracking), and (6) group coordination


Other aspects of foraging might push children to pursue one resource rather than another. As mentioned, they might pick up a task normally performed by another member of the family that becomes absent. But, also, children in different settings may be differently motivated (by hunger, play, prestige, etc.) to pursue different types of resources at different ages. Processing time is another important aspect of choosing what and when to collect, and not always the person who collects the food is the same who processes it, making it difficult to interpret foraging returns at the individual level.  


%THOUGHTS
%How to infer selective pressures (e.g. the pressure to delay reproduction in order to improve foraging returns through skill acquisition) vs proximate causes such as reasons why individuals would forage in a specific moment. Also, trading off between future and current returns (foraging simpler things, for more returns, but also learning for future, higher, returns over more complex things - i.e. transfer of competencies across preys?)
 
%meta-analysis we aim to summarize previous research both qualitatively and quantitatively. Starting from a cross culturl analysis published by \cite{koster_life_2019}, we developed a model to analyze published data on children foraging returns. 

%Children foraging activities interest different areas of the study of human behavioral ecology. The hunting and gathering of children and adolescents has a role in household economy and division of labor studies, cooperative breeding in humans and in investigating the evolution of life history traits. It is surprising, then, that data is scarce and a unifying framework is missing. In this paper, we compare the available data on children foraging returns in the current literature and argue that: - children forage extensively in several societies, - differences in productivity can be explained by ecological contingencies, - the improvements with age vary across culture and task. We also address the context in which research on children foraging has been conducted and call for an increase in data collection as well as for sharing protocols and methods. A concerted effort to create datasets comparable across cultures would allow to address important questions of human evolution.

%Contrary to what happens in western societies, children in many other parts of the world contribute substantially to their own maintenance. Participation of prereproductive individuals to foraging has been documented ethnographically in a large number of populations, ranging in their subsistence strategy from hunting and gathering to market integrated commerce, but the extent and returns of foraging activities have not been compared across cultures. 

INTRODUCTION
Second version
%A puzzling  aspect of human life history is our extended childhood. As a period in which offspring is still dependent from adults, despite more active and autonomous than an infant, childhood is essentially unique to humans (\cite{bogin_evolutionary_1997}). 
%On the one hand, children appear to be net consumers through adolescence (\cite{kaplan_theory_1996}). This prolonged period of dependence is hypothesized to have evolved to facilitate the acquisition of the embodied capital necessary to successfully extract nutrient-dense resources from our foraging niche  (\cite{kaplan_theory_2000}).
%On the other hand, children are active helpers (\cite{kramer_childrens_2005}), specializing in foraging tasks matched to their size, skill, and strength. In some cases, children’s foraging efforts offset a considerable portion of their caloric needs (\cite{bird_children_1995, kramer_variation_2002, crittenden_juvenile_2013}). 
%In order to resolve this paradox, human evolutionary scientists have investigated the age-specific foraging returns of children in small-scale subsistence societies. While informative, these studies are almost exclusively limited to single populations, making it difficult to uncover how social and ecological environments enable and constrain children’s foraging participation across cultures. The present paper presents the first meta-analysis of age-dependent variation in children’s foraging returns. In what follows, we outline some of the key theoretical debates and empirical findings regarding the evolution of childhood and children’s foraging returns. We then outline the results of our meta-analysis, which show that [...]. We conclude by arguing [...].

First version
%All around the globe, children search their environment looking for and collecting food. 
%They can be berry picking on summers, or consistently collect half their caloric needs: no matter how much their diet is composed of wild food, children forage. 
%Information on how, when and how much young individuals forage is a contribution to different disciplines studying human behavior.

%For instance, if we want to understand how humans make foraging decisions over cost-benefit trade offs in terms of energy, time, social status, etc., the specific constraints and needs of children and teenagers can offer a variety of case studies (\cite{bliege_bird_children_1995, bird_ethnoarchaeology_2000, bird_children_2002}).
%Or, in a more general application, the study of nutritional status and requirements in various ecological settings needs to take into account what food children and adolescents produce, as this can be an important component not only of their diet (\cite{pollom_changes_2020, pollom_effects_2020, mcgarry_children_2009}),but also of that of their families (\cite{fouts_who_2009}). 

%The fact that young individuals have a double role as producers, as well as consumers, has obvious implications for household economies, which can count on incoming food from each child, as well as for reproductive decisions of their parents, which can rely on their offspring feeding their siblings when performing childcare duties (\cite{kramer_variation_2002, kramer_maya_2005, kramer_childrens_2005, kramer_does_2009, crittenden_juvenile_2013}). 
%As a consequence of being interrelated at these multiple levels with human behavior, children foraging activities have probably contributed to the evolution of several modern humans life history traits. In order to learn the complex set of skills required to reach full foraging potential, natural selection might have promoted an extension of the pre-reproductive period in a prolonged condition of at least partial dependency (\cite{kaplan_theory_2000}) and collecting data on children foraging in fundamental to test this hypothesis.
%At the same time, the fact that in most cases the dependency of children is not complete, as they procure part of their own food and contribute to household workload, but decreases in time, is important to permit the evolution of the characteristically human trait of having overlapping dependent offspring (\cite{kramer_childrens_2005}). Crucially, this allows humans to reach incredibly high fertility rates for an ape.%cite

%Finally, as foraging children carry on traditional subsistence practices while their parents engage in market labor elsewhere,they can engage in the acquisition, and thus preservation, of traditional ecological knowledge (\cite{setalaphruk_childrens_2007}).

%Given the applied nature of these information, tracking how knowledge is acquired and used can give important insight on learning and teaching strategies in traditional societies, as well as on psychological mechanisms underlying these processes (\cite{lew-levy_who_2019}).

%INPUT
%Did I forget anything fundamental?
%More on psychology, archaeology or on diet, or environmental change, effect of schooling?
%of course we could go on forever, but feel free to signal if I missed something you deem relevant.

%Despite the relevance of this behavior for different fields, relatively few researchers have collected return data from children foraging. Comparative works are especially lacking and, as a result, little can be said on general trends. 

%\vspace{0.5cm}
%With this paper, we aim, first, at providing an historical and thematic overview of researches describing children foraging activities. 
%Specifically, we review how the perception of children foraging changed since the phenomenon has been first described quantitatively; we summarize the main contributions of research on children foraging results and describe some of the main papers that report this kind of data.

%Second, we provide a comparative analysis of foraging returns across \Sexpr{nlevels(unique(metadata[,"Population"]))} different societies, tapping on previously published data available in the literature. 
%We analyse how trajectories of foraging returns change with age, and compare different resource types. To do so, we developed a Bayesian model inspired by \cite{koster_life_2020} which allows to understand how the factors determining increasing returns with age contribute to the total variation. 
%To address the difficulties of comparing different data sets, within we focus on the change of foraging returns of children and adolescents (with an arbitrary cut off at 20 years of age), relative to comparable adult values.

%Finally, we focus on one hypothesis tested in several of the analyzed papers, that human early life history has been modeled to allow for learning of complex skills \cite{kaplan_theory_2000}, and look for correspondences on both the data and the conclusions drawn from them across papers.  

Foraging and diet of children - quotes
%\cite{mcgarry_children_2009}However, 62% of all the children interviewed (both non-school and school-going) supplemented their diets with wild foods. Dietary diversity showed a 13% increase when wild-food supplementation occurred. 
%\cite{fouts_who_2009} "Child feeding involvement by juvenile relatives (e.g., siblings and cousins) was not predicted by family transition, birth order, or physical ecology. However, it is striking to note that juvenile relatives provided food as much as adult female relatives, and nearly as much as elderly female relatives. Since birth order did not predict levels of child feeding by juvenile relatives, one could assume that even if a child is first born, with no older siblings, they are still receiving care from cousins. This is not surprising since the Aka live in large extended family groups and juveniles typically make up at least half of the population. However, the lack of birth order effect is somewhat contradictory to evolutionary hypotheses (see Turke, 1988) that suggest siblings in particular will provide care due to their close genetic relatedness compared to other alloparents with more distant genetic relatedness (e.g., cousins)." 
%Borgerson, Cortni, et al. "An evaluation of the interactions among household economies, human health, and wildlife hunting in the Lac Alaotra wetland complex of Madagascar." Madagascar Conservation & Development 13.1 (2018): 25-33. for indication that supplementing diet could reduce usage of wild foods

Foraging and knowledge
%For example, \cite{chpeniuk} observed that individuals who foraged a higher breath of goods have a better sense of biodiversity as adults.
%studies on how children interpret and conceptualize the natural environment often contrast data from groups where children have different levels of reliance on wild foods.
%cite paper on hunting in the us, and have a look at katja leibal and annie wertz work

Expansion of introduction
%effect of changing environments: if children foraging is important, and an environmental change makes it impossible, is there an effect on health? 
%Also, schooling: if children go to school and do not hunt, for example, how is this supplemented in their diet? Schools provide meals, but sometimes cheap carbs (Tanzania liquid porridge)

%info for archaeology and other fields

Other introduction drafts

%BIN

%Trends of  foraging return increases with age relative to average adult value have been modeled taking into account variation in resource collected, society of provenance and method of data collection. 
%Foraging requires several skills and imposes complex trade offs, that inevitably change as children grow and their bodies change. 

%foraging important for HBE and developmental trajectories interesting
%Foraging activities have been a central theme in Human Behavioral Ecology since its conception (\cite{laland_sense_2011}, chapter 4). This is due, in one hand, to the assumption that foraging returns are associated with reproductive fitness, which makes them relevant for evolutionary studies ("Consider a situation in which there is a direct relationship between time spent hunting, the amount of game acquired (measured in total weight or its equivalent in calories or nutrients), and the relative reproductive success of the hunter." in \cite{hawkes_how_1985}). And, in the other hand, the fact that returns can be clearly quantified put them at the center of several models producing testable predictions, as in optimal foraging theory (for a review, see \cite{smith_anthropological_1983}). 

%Later, researchers started focusing on the development of foraging skills, bringing attention to age specific returns. For example, \cite{koster_life_2019} developed a model that traces age trajectories of individual hunting returns in 40 study sites. 

%At the same time, the work of young individuals in subsistence societies became apparent through quantitative and qualitative analyses.
 
 
%why is the analysis of children behavioral returns important (evolution of life history, learning and mechanisms limiting returns, children welfare -e.g. nutritional status, social role and schooling-, impact of ecological and social change, all need info on children work, role in the family and also what and how much they forage.) 
%Understanding how children and adolescents interact with their environment to extract food resources has implications in various areas of the study of human behavior. 
%For example, nutritional status and children welfare are directly affected by their foraging activities: is the food they procure an important contribution to their caloric intake? Are the socializing opportunities that happen during this activity important for the development of individuals? The effects of social innovations, such as the increase in schooling time, or the ecological changes can have important impacts on children nutritional status. 
%Also, Being a proficient adult in any human society requires a vast array of skills, that are necessarily learnt over time. How returns change along the life of individuals can help clarify how this process of skill acquisition works.

%in order to do this, we need to better understand how children improve their foraging returns, and what contributes to this change, and also understand if there are general trends 


REVIEW
Old version
\section{Foraging for children research: a review}

We review here the existing literature concerning juvenile foraging activities, adopting a combination of historical and thematic approach. We start by describing how children foraging has come to the attention of researchers during the second half of the 20th century. We then follow by discussing some of the main subjects interested by data on children foraging, highlighting the areas in which more quantitative information would be important, for example to answer question about human evolution or to design policies in changing environments.

\subsection{Becoming adult}

Despite ethnographic or descriptive account of children foraging, quantitative research on these activities had a very slow development. %INPUT cite Mead and others?

Among the first works focusing on children in small scale societies, \cite{lee_social_1976} strikes for how it depicts children as economically dependent from their families. Collecting time allocation data among the !Kung hunter gatherers of the Kalahari desert, she deduces that young individuals barely engage in productive activities independently from their parents. This is explained in part with the technologies available, and in part with the distribution of resources in the environment. The technologies !Kung employ to forage, according to Draper, are too simple and do not facilitate difficult tasks enough for children to engage with them. Also, low predictability in time and space of food resources contribute to excluding children from productive activities out of camp. 

In the following decade, \cite{nag_anthropological_1978} and \cite{munroe_childrens_1984} (among others?) also measure time allocation of children. Focusing on village and domestic settings, these papers offer a slightly different account of children work. In the societies they observe, children as young as three years old are reported spending at least a part of their time in productive/household activities. These include childcare and garden cultivation, but remarkably not foraging. Such an absence might be due to the fact that the studies are conducted in agricultural societies ('paesant societies' according to Nag et al.), but also to the method of data collection, that appear not to allow for observation in forests or other places where children would forage.

Up to the late '80s, the only paper reporting reporting quantitative accounts of children foraging is \cite{kawabe_development_1983}. This paper, though, which describes Gidra boys hunting and fishing in the lowlands of Papua New Guinea, goes almost unnoticed in the contemporary literature. 

\subsection{Hadza vs !Kung: renewing the approach on children foraging returns}

It is understandable, then, that researchers approach children foraging activities expecting low engagement in this activity. For example, \cite{blurton_jones_why_1997}, titled "Why Do Hadza Children Forage?", list different hypothesis of why indeed children would forage in a hunter gatherer population. 

In particular, in the literature of the early '90s, the question of why would children take up foraging is framed as a comparison between the !Kung and another African hunter gatherer population, the Hadza. Contrary to !Kung, Hadza children venture in the scrubland surrounding their villages in small mixed age groups to dig tubers, hunt small game or pick fruits.  \cite{hawkes_hadza_1995}, p. 694, for instance, states: "The Hadza are an exception to the conventional anthropological wisdom that the children of mobile hunter-gatherers do little to support themselves but instead rely completely on their parents for subsistence". 

The contrast is sharpened by the fact that !Kung and Hadza are similar under various aspects. Considered remaining examples of the hunter gatherers inhabiting Eastern and Southern Africa before the Bantu expansions, both populations largely rely -or relied until recently- on wild products for food and shelter. They live in mobile camps following the availability of water and other resources, and have traditional diets comprising mainly tubers, nuts and fruits and wild meat (citation needed).

But the differences, in this case, count more than the similarities.
\cite{blurton_jones_foraging_1994, blurton_jones_differences_1994} describe the different hardships offered to Hadza and !Kung by their environment. Despite both populations live in arid, Savannah-like climate, Northern Tanzania, where the Hadza live, is a more favorable habitat. The landscape is more varied, reducing the chances for inexperienced children to get lost, and the risk offered by predators or other adversities is considered lower by Hadza parents. But the main factor seems to be the distance of resource patches from camp. Hadza children have usually access to baobab trees within few hundred meters from camp. When the trees produce fruits, the children can collect up to half of their daily caloric needs by picking them. In order to collect a comparable amount of calories, !Kung children should cover more than five kilometers to the closest Dobe nut grove. For the sake of comparison, no Hadza child goes that far from camp without adult supervision. Moreover, children do not follow their parent in the excursions to these groves, as adults report that the lack of water and shade along the way makes children more of a burden than help. 

In the same paper, Blurton Jones and colleagues suggest that for children to accompany the adults on these long trips is actually sub-optimal at the household level. Children appear to spend their time more productively by processing the food collected by their mothers, rather than collecting it themselves. More in detail, \cite{hawkes_hadza_1995} calculate the costs and benefits of different tasks children could accomplish, comparing data on various activities payoffs for the mother-offspring team. Beginning "with the simple hypothesis that children seek to maximize their mean rate of nutrient acquisition while foraging" (p. 694), they find that "Just as Hadza mothers earned higher team rates by taking their children to the distant berry patches than by foraging closer to camp, so !Kung mothers earn higher team rates by not taking their children to the mongongo groves, leaving them at home to crack nuts instead" (p. 697).

From this in comparative analysis, emerges as the main message that children can behave optimally and that what is optimal depends on the environment and resources available. The work on juvenile foraging that follows tries to increase the amount and variability of data available, to understand how children and teenagers face cost-benefit trade offs in different conditions, as well as the evolutionary implications to foraging behaviors.

\subsection{A children's world}

In particular, in the early 2000s, Rebecca Bliege Bird and her husband Douglas Bird published a series of papers on children foraging activities. For the first time, what children do was observed from their own point of view: "Paying attention to the differences from a child's perspective may help adults remember what being a child was all about." (\cite{bird_children_2002}, p292).

Working first on the island of Mer, in the Torres Strait Archipelago, and later on among the Mardu of Western Australia, the couple tried to understand the factors defining the trade offs children face when foraging. In both societies, children leave often the camp to hunt or fish, and are often collecting a large proportion of their daily caloric need ("Meriam children under age 15 each provided an average of 750 kcal/day", \cite{bliege_bird_children_1995}, p 16).

In \cite{bliege_bird_children_1995} and later more in detail in \cite{bird_ethnoarchaeology_2000} and \cite{bird_children_2002}, the authors observe how children and adult foraging behaviors differ on the intertidal reefs of Mer Island. They argue that the difference in the proportion of shell species collected is a function of adult vs. child body size, and that this difference is consistent with predictions from prey-choice models. 
%These models predict whether or not a prey should be pursued on encounter. They are used to understand decision making during foraging, because some preys carry a premium lower than the cost paid to capture/process/transport that prey, and thus are not worth collecting. Changes in diet composition would imply, according to this model, a change in the cost benefit ratio of certain species, or reduction in the abundance of others. In this case,
Specifically, children are smaller and walk at lower speed along the reef, which reduces their chance of encountering high value prey. It becomes optimal, then, to include lower value preys in the diet, and hence we observe the appearance of the small \textit{Trochus} and \textit{Strombus} in the children bounty, which are not collected by the adults.
%The \cite{bird_ethnoarchaeology_2000} paper is especially interesting, as it explicitly tests predictions from the prey choice model, predictions which are largely confirmed by real data. This has implications for the interpretation of archaeological data, as similar patterns could be originated as an effect of children foraging, as an alternative to other mechanisms such as a reduction in the abundance of certain species.
Similarly, in \cite{bird_mardu_2005}, Mardu children chose how to spend their foraging time in accordance with their size. Being slower than their parents, it is advantageous for them to hunt the smaller but more common lizards which live in rocky patches close to their village, instead of traveling to further patches where adults hunt. This way, they increase their total returns, by both reducing energy and time spent in traveling, and encountering preys at a higher rate 
%(as well as reducing the risk of dehydration or heat strokes and making time to care for their younger siblings). 
Other factors defining what children forage might have nothing to do with their strength or size, but instead with what adults choose to prioritize. Social status or apparent reproductive value have different fitness consequences for parents and offspring, translating ultimately in different decisions in front of trade offs (\cite{bird_constraints_2002}). This has been used to explain big game hunting, which in some hunter gatherer societies has lower return rates than other less prestigious activities, but is still performed by individuals trying to increase their social capital or to impress a potential mate (citation needed \cite{}).   

By paying attention to children foraging, then, we can not only better understand the factors directing choices between trading off options, but also test models regarding foraging behaviors under a variety of assumptions.

\subsection{Evolution of childhood}
During the '90s and early 2000, new attention to children foraging returns has been brought by a growing literature discussing the evolution of childhood (\cite{bogin_evolutionary_1997}).%more citations.

In particular, Kaplan and colleagues developed a theoretical framework which would explain the emergence of several aspects characterizing human life history as a coevolution with 'dietary shift toward high-quality, nutrient-dense, and difficult-to-acquire food resources.' (\cite{kaplan_theory_2000} p.156, and see \cite{kaplan_theory_1996, kaplan_evolution_1997, kaplan_embodied_2001, kaplan_emergence_2002, kaplan_embodied_2003, kaplan_neural_2003, kaplan_life_2006, kaplan_evolution_2007}). %Namely, 'an exceptionally long lifespan, an extended period of juvenile dependence, support of reproduction by older postreproductive individuals, and male support of reproduction through the provisioning of females and their offspring' would have been selected to allow the exploitation of a highly complex foraging niche. High return food packages, such as big game, but also tubers and hard-to-get resources, would allow to support dependent offspring, but also require the development of foraging abilities over a prolonged period of time. 
This framework, often referred to as the Embodied Capital Hypothesis, implies that individuals increase their capacity to extract resources from the environment as they build up the set of traits that allows them to do so (skills, strength, knowledge etc, in general referred to as embodied capital, from human capital theory in economics), until when they are able not only to collect enough food for their own energetic requirements, but also to produce a net energetic surplus to feed their dependent children or even grandchildren. To test this hypothesis, then, data on how food production changes along the lifespan must be collected, with a focus on the pre-reproductive period.
%"First, high levels of knowledge, skill, coordination, and strength are required to exploit the suite of highquality, difficult-to-acquire resources humans consume."

%In \cite{kaplan_evolution_1997}, to exemplify the hypothesis, the average productivity per age is compared with the consumption for three different societies. But with more attention to individual level returns,
In the early 2000s, several authors working on children behavior finally set up to doing so. 

A special issue of Human Nature, published in June 2002, reports discussions on the evolution of childhood with data from four different populations. \cite{bird_children_2002} collected naturalistic return data from children collecting seashells on the island of Mer, and also, in \cite{bird_constraints_2002}, from young Mardu hunging lizards in the Western Australian desert. \cite{bock_learning_2002} observed and measured both food collection and processing in a multi-ethnic community in the Okavango Delta of northwestern Botswana. And, finally, \cite{blurton_jones_selection_2002} organized competitions to measure the development of a combination of foraging-relevant techniques among the Hadza in Tanzania. 
Also in 2002, \cite{walker_age-dependency_2002} focused on hunting returns among the Ache of Paraguay with a combination of naturalistic observations, contests and recording of activities on a diary by informants. 
In 2005, \cite{bock_what_2005} published again on the same subject, and \cite{tucker_growing_2005} contributed with data on tuber digging among the Mikea hunter gatherers of Madagascar. 
\cite{gurven_how_2006}, combining interviews and naturalistic observations, reported hunting returns among the young and adult Tsimane horticulturalists of Bolivia and, most recently, \cite{crittenden_juvenile_2013} described children foraging for various products among the Hadza with naturalistic observations of returns.

The majority of these papers shares some similar features. 
First, they set up to discuss different hypotheses that can explain the evolution of childhood, often contrasting the Embodied Capital Hypothesis with some other possible mechanism underlying the elongation of pre-reproductive period in humans. 

Second, they often remark on the complementary contributions to foraging success of two types of embodied capital. Physical abilities, such as strength, coordination or speed, depend on the development of the body and constitute a growth-based capital, while knowledge, skill and intuition are an experience-based capital, which requires time and implementation to be acquired. 
The data are deemed to hold different patterns according to the alternative hypotheses on how they are generated and to the interpretation the authors make of these hypotheses. For example, time spent in school, depriving children of the opportunity to practice, should have the effect of reducing returns depending on experience-based capital (\cite{blurton_jones_selection_2002}).
The conclusions these papers reach are not consistent, but some general messages can be extracted. 
%AND WE WILL TREAT THE IMPLICATIONS IN GREATER DETAIL IN THE THIRD SECTION OF THIS PAPER.

Mainly, the way returns change with age depends on the specific constraints of the activity taken into consideration. For example, hunting appears to be a particularly skill intensive activity, with multiple combining elements each depending on different experience-based forms of capital (finding prey, tracking, archery skills), but also requires some strength and endurance. The data suggest that hunting might be one of the foraging activities for which returns peak later in life. 

Hunting is treated remarkably clearly by \cite{koster_life_2020}, with a large cross cultural database of juvenile and adult returns for more than 1,800 individuals from 40 societies, spanning several years. Age at peak hunting success is in general quite high, with an average between 30 and 35 years of age, but it also varies a lot across societies. Environmental variability in the nature as well as the frequency of encounter with prey might be at the root of the differences, further reinforcing the idea that foraging success depends on the specific condition of the task.
%INPUT maybe we should expand on Jeremy and Richard's paper, or we do more when describing the model?

\subsection{Cooperative breeding, pacing of reproduction and self sufficiency}

Children productive activities have repercussions on other aspects of human life history. \cite{hawkes_hadza_1995} demonstrated that the dyad mother-offspring maximizes joint returns when foraging together. On a higher level, children production has consequences for the whole household, reducing the cost of rearing each child through self-support and often also contributing to feeding younger siblings when adults are absent. Although in most societies individuals don't reach self sufficiency until when they are in their late teens or early twenties (\cite{kaplan_evolution_1997}), and only later on produce enough surplus to support a family, the fact that children forage might allow the peculiarly human pattern of reproduction with multiple dependent offspring. This subject has been widely explored with time allocation data in \cite{kramer_variation_2002, kramer_maya_2005, kramer_childrens_2005, kramer_does_2009} and elsewhere, and also partially discussed in \cite{crittenden_juvenile_2013} and \cite{bird_constraints_2002}, but would deserve more attention with the collection of individual and family level data on production and consumption. 

\subsection{Learning and knowledge}
A different area of research that has demonstrated interest for children foraging activities is the one studying learning and development of knowledge in small scale societies (\cite{gallois_local_2017, koster_wisdom_2016, lew-levy_how_2017, lew-levy_who_2019, reyes-garcia_adaptive_2016, setalaphruk_childrens_2007}). Adult individuals need a vast array of information to navigate their social and ecological environment. A good part of this knowledge is relevant for foraging, so that a big part of the field focuses on understanding the transmission and measuring ecological knowledge at different ages. Investigating the role of environmental information for practical purposes, \cite{koster_wisdom_2016} measures knowledge relevant for fishing activities in individuals of at least 10 years of age in a Mayanga village in Nicaragua. He also looks for, and fails to find, correlation between these measures and estimated fishing abilities. \cite{reyes-garcia_adaptive_2016} on the contrary finds out that local ecological knowledge provides individual returns in terms of hunting yields. In light of the increasing interest for local knowledge alongside with learning strategies in small scale societies, more work on how children employ the acquired knowledge when foraging would be very useful. 

\subsection{Diet}
By foraging autonomously, children have the potential to integrate the diet they receive at home. In poorer and marginalized settings, especially, the contribution of foraged wild foods can be important (add citations) and children can procure these foods by themselves, even as side consequence of playing activities. Moreover, individuals, as they grow, often have dietary needs that differ from those of the adults and might include foods that are not appreciated or provided by caregivers (citation search things on unripe mangoes for example) which they could have access by searching individually.  
Despite the potential relevance of wild foods in nutritional status, the data on children involvement in foraging activities are scant in non hunter-gatherer settings. 
\cite{mcgarry_children_2009}, as an example, observe an increase in children's diet quality and variability thanks to foraged food in rural South Africa, especially in poorer households. On the contrary, \cite{lee_childrens_2009} does not find associations between child foraging and nutritional benefits in a Mexican shanty town. 
Studying children contribution to diet, is particularly important in societies undergoing dietary transitions. A number of health concerns is associated with changes in availability of agricultural and then processed foods (\cite{satia_dietary_2010}). Focusing on how transition to domesticate foods affect foraging patterns of Hadza children, \cite{pollom_changes_2020} report a reduction in the variety of wild foods collected between 2005 and 2017. 

The implications of children and teenagers foraging activities for the development of policies to improve health and education in marginalized areas would call for more effort in these kind of studies.

\subsection{Emergence of sexual division of labor}
Finally, observing children foraging activities is necessary to understand the emergence of sexual division of labor. In many hunter gatherer populations, the two sexes target different foods when foraging as adults, but have similar foraging patterns as children. We can get some insight from time allocation data, which show females devolving more time to household and childcare tasks as they age, whereas males spend more time further from home (\cite{froehle_physical_2019}, and Lew-Levy (hopefully in press)). But specific analyses trying to understand when the gender specialization in foraging emerges are missing.

\subsection{Conclusions}
The widespread prevalence and extensive implications of children foraging behaviors would call for more attention to this phenomenon. Children based research raises multiple specific concerns, not least the ethical implications of working with minors, but the possible benefits, especially because concerning marginalized societies, can outnumber the difficulties, as long as research is appropriately conducted.

In the following section, we use data coming from many of the studies cited above to delineate a general profile of children foraging behavior. We hope this can be used as a background for future research expanding on the themes presented here. We also provide some indications to make children foraging research more comparable and better at contributing to variety of scientific questions.


%INPUT please, check these sections out, if you think an extension of any would be advisable, or sections are missing, you can proceed or let me know and I'll do my best. Comment out passages if you think it would be advisable to shorten the text.
\subsection{Hadza vs !Kung: renewing the approach on children foraging returns}
%focused on two African hunter gatherer populations: the !Kung and the Hadza. The peculiarity if this case study is that, despite many similarities in lifestyle and Savannah-like environment, children behave very differently in the two groups. While the !Kung children spend most of their time in the village and collect almost none of the food they eat, Hadza children are very active, move freely in and around the camp and routinely contribute to their own maintenance. According to \cite{hawkes_hadza_1995}, both populations display an adaptive behavior: among the !Kung, the dyad mother-child obtains higher returns when children perform processing tasks at the village, while mothers collect food, which is present only far from camp. On the contrary, Hadza mothers routinely bring their offspring along during foraging excursions to certain types of patches, maximizing total returns according to age-specific return rates for each resource. 
%(\cite{hawkes_hadza_1995}, p. 694: "We begin with the simple hypothesis that children seek to maximize their mean rate of nutrient acquisition while foraging."
%p. 697: "Just as Hadza mothers earned higher team rates by taking their children to the distant berry patches than by foraging closer to camp, so !Kung mothers earn higher team rates by not taking their children to the mongongo groves, leaving them at home to crack nuts instead. The difference lies in the processing requirements of the resources that provide the highest available team rates."
%p. 698: "The foraging patterns of Hadza children are determined by the age-specific return rates for local resources. Comparisons with the !Kung show that the processing requirements of these resources are also determinant."
%p. 699: "Two variables emerge as important in explaining the character of children's productive activity among mobile foragers. The first is the age-specific return rates for locally available resources that determine which alternatives give the highest team rates for women and children. The second is the character of the resources offering the highest team rates.")

\subsection{A children's world}
%\cite{bliege_bird_children_1995}"We hope, through our continuing research on Mer, to understand w h y children would allocate time to foraging or other productive activities, and to understand what might account for the differences between children's and adults' foraging strategies." p. 3
%"Children of the Hadza in east Africa, for example, provide a daily average of 614 kcal towards their own subsistence (Blurton Jones et al 1989), while Ache children of Paraguay, under age 18, supply an average of 557 kcal/day, Machiguenga children in Peru produce 495 kcal/day, and, among the Piro of Peru, children produce 366 kcal/day (Kaplan nd). In comparison, in our non-randomly selected sample, Meriam children under age 15 each provided an average of 750 kcal/day, for either themselves or their families, in addition to an unknown amount gathered opportunistically in the village. Their return rates are based on lowest estimations of caloric content and are probably even higher than indicated in Table 1, but again the non-random nature of the sample suggests this figure is at the upper end for the Meriam children as a whole." p 16

%\cite{bird_children_2002} "Here we investigate whether differences in the prey choice of Meriam children and adults while reef-flat collecting are a result of attempts by children to learn adult strategies, or whether their efforts reflect differences in the constraints the children face while foraging on the reef." p 272
%"Paying attention to the differences from a child's perspective may help adults remember what being a child was all about." p292



\subsection{Evolution of childhood}
% Presented by Kaplan in some early papers and then more extensively in the 2000 paper
%     Human success is due to the exploitation of a complex foraging niche
%     Human Life history is an adaptation to it
%     Childhood because we need to get good at exploiting that niche (learning, growing, skills)
% This allowed to make predictions, that would be tested with data on children foraging
% In early 2000 several researchers collected data on children foraging to test Kaplan’ hypothesis
% Methods and conclusions vary, but they observed that children forage and that they get better with age
% But not homogeneously, there are differences between sexes and resources
% Especially hunting seems to peak late -walker, koster, discuss skill intensity of foraging, maybe gender differences

%\cite{bird_children_2002} "Because a number of phenotypic traits are correlated with age, including size, experience, digestive morphology, and nutritional requirements, one cannot use simple age differences in behavior to support the hypothesis that a long period of juvenility is a response to the complexity of adult behavior."p292

\subsection{Cooperative breeding, pacing of reproduction and self sufficiency}
%Maybe not relevant here, not many papers with return data in this section, but discussed in the intro and definitely an area where where expansion would be ideal...

%\cite{crittenden_juvenile_2013} Juveniles are both dependents and providers. This “two-fold” nature of human juvenility is an often overlooked, yet critical, dimension of human life history and the evolution of cooperative breeding (Kramer, 2011). Themajority of life history models evaluating juvenile contributionsmeasure net economic value, yet young foragers are concurrently collecting, exchanging, and receiving resources before they become a net producer (Kramer, 2011). Hadza juvenile foragers are no exception; although not collecting the entirety of their daily caloric requirements, they aremaking substantial contributions byway of self-provisioning. The caloric contributions provided by juveniles underwrite the cost of their care and may contribute to a mother's ability to successfully raise multiple dependents, thus supporting the notion of humans as cooperative breeders.

%\cite{kramer_maya_2005}
%The timing of the transition from net consumption to net production shapes the age range of children dependent on others ends
%If the span of time between when children become economically independent and when they are no longer available as keepers is variable across cultures, what conditions that variation is of interest because of its effect on the competing demands of parents having to support the disparate needs of both younger and older children and, ultimately, on the number of children parents can successfully raise.
%Given these concerns, neither money nor calories adequately reflect work effort in the Maya case. Consequently, time was selected as the most suitable currency for measuring an individual’s consumption and production.

%\cite{kramer_variation_2002}
%Maya children produce more than they consume by their early to mid teens but remain in their natal households for a number of years before leaving home and beginning families of their own. The Maya results contrast markedly with those from several groups of hunter-gatherers and horticulturists for whom we have similar data. Even in the Maya case, where children are self-sufficient at a relatively young age, parents are unable to support their children without help from others. The production surplus of older children appears to help underwrite the cost of large Maya families and subsidize their parents' continued reproduction.
%The timing of children's economic independence influences the parents' time and resource budgets in two important ways. First, in most animal species, juveniles become self-sufficient before or at sexual maturity, but among humans this may or may not be the case. When older children are able to support themselves, parents have a greater available budget to care for younger children if they so choose. Second, the production surplus of older children can be transferred to help subsidize dependents, allowing parents to raise more children than they might otherwise be able to do.


\subsection{Diet}
%\cite{lee_childrens_2009} does not find associations between child foraging and nutritional benefits in a Mexican shanty town, whereas \cite{mcgarry_children_2009} observe an increase in diet quality and variability thanks to foraged food in rural South Africa, especially in poorer households. Changing dietary patterns in hunter gatherers populations have an effect on the foraging habits of juveniles.

%The foraging activities of children have impacts on their diet. Outside of hunter gatherer populations, children have been observed foraging in marginalized urban (\cite{lee_childrens_2009}) and rural environments (\cite{mcgarry_children_2009}). Despite the growing body of evidence indicating that wild foods are beneficial for diet quality (add citations), the studies focusing on the impact of foraging activities on health status of children in precariours nutritional status are surprisingly few. 

More from review

% On the side and especially in the last decade, other areas of research emerged that took an interest in children activities and foraging, although not all of them actually report foraging returns (many consider time allocation, which is no good for our meta-analysis, but interesting for theory)
% Many studies demonstrate that children forage consistently and contribute to a substantial amount of their caloric needs. This has implications for the understanding of different phases of life history, including the length of the pre reproductive period and inter birth interval. 
% Some studies started to have an interest in children foraging in other environment, such as urban
% Or in a context of environmental and social change, as Draper did first in 1988- Pollom 2020 on effects of ecological change on subsistence have to consider children
% Or finally there’s more attention on how children learn, which is tangential to the theme of foraging, trying to explain variation
%patterns of food sharing emerge when
%learning


%Early studies in the field of Human Behavioral Ecology (HBE) paid little attention to children. The focus on hunting, especially big fauna, as main form of foraging and the high relevance in the HBE literature of a population in which children exhibit a particularly low levels of activity, the !Kung, contributed to this low level of interest. 

%One of the earliest studies focusing on children activities in the !Kung \cite{lee_social_1976} describes the various factors limiting the possible contributions of children to food production. The dangerous environment, a simple bur somewhat unhelpful technology, the unpredictability of resources, all result in 'the absence of children work'.
