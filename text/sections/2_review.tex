\section{Foraging for children research: a review}

\subsection{First Studies: San and Hadza}
Despite descriptive accounts of children's foraging in many earlier ethnographic reports, the first quantitative  studies on this topic were conducted only in the late 1980s and 1990s by Blurtion Jones, Hawkes and colleagues (\cite{blurton_jones_modelling_1989, blurton_jones_differences_1994, blurton_jones_foraging_1994, blurton_jones_why_1997, hawkes_hadza_1995}).
Comparing the foraging returns of Kalahari San and Tanzanian Hadza, these studies examined how ecology shaped children’s foraging participation. Specifically, San children collected almost none of their own food resources, while Hadza children actively contributed to their own subsistence. Despite some similarities in the environment, differences in water availability, risk of predation and the proximity of resources to settlements appear to have affected how San and Hadza children and their parents coordinated their labour to maximize foraging returns (\cite{hawkes_hadza_1995}). For example, San mother-child dyads produced more kcal/hour  when children remained in camp to process \textit{mongongo} nuts than if they helped their mothers collect them. Hadza children, on the other hand, routinely accompanied their mothers to predictable berry patches, which maximized mother-child total yield. These early studies were the first to show that children’s foraging should be considered in relation to their ecological and family context, and that children could make substantial contributions to foraging returns, even if they would not always do so. To our knowledge, however, these were the only series of studies to systematically compare children’s foraging returns across cultures. 

\subsection{Optimal Foraging}
That children and adults cooperate to maximize foraging returns is one aspect of Optimal Foraging Theory (OFT). OFT investigates how animals behave when searching for food, under the premise that foraging efficiency translates to biological fitness, and thus, has been maximized by natural selection (\cite{pyke_optimal_1977}). Pioneered by \cite{macarthur_optimal_1966}, modeling the optimal use of patchy environments by an ideal predator, OFT has been successfully applied to foraging data in human populations  (see \cite{smith_anthropological_1983, hawkes_optimal_1992, winterhalder_hunter-gatherer_1981}). These studies have shed light on important aspects of human behaviour, including how much time foragers should spend in a patch of resources before moving to the next one in order to maximize returns (Patch Choice models), or where camps should be located to facilitate access to scattered resources (Central Foraging models). 
Particularly important in the human literature has been the Prey Choice model (also known as Diet Breath model), which predicts the foods that should be included in the diet given the cost of pursuit over the benefits derived from consumption. 
%\textsc{
These studies measured time and/or energy invested in finding, capturing and processing each type of animal or plant food, and their calories or nutrient composition. If a certain resource gives a return per unit time that is higher than the average return of all the other resources in the environment, it will be pursued, otherwise it will be ignored in favor of some better prey.
%} 
%\textit{These studies demonstrated that foragers coordinate their labour…..men and women face difference trade-offs and thus target difference resources… Flesh this out in 1-2 sentences to set up the next para}

Several authors tested OFT expectations on human data, focusing largely on adults (\cite{hill_foraging_1987} ). Much less research has focused on how children optimize their foraging returns. Bliege \cite{bird_children_1995, bird_ethnoarchaeology_2000, bird_constraints_2002, bird_children_2002, bird_mardu_2005}, working with Australian Meriam and Mardu people, observed that foraging children target preys matched to their size and strength. For example, Mardu children’s walking speed restricted their hunting success on sandhills. As a result, children focused their hunting efforts on lizards in rocky outcrops, where they encounter prey more quickly and with less walking. 
Subsequent research by \cite{tucker_growing_2005} among Malagasy Mikea showed that children focused on collecting younger and shallower tubers in patches abandoned by adults. Unlike adults, Mikea children did not strive for efficiency: Tucker observed a ‘food fight’ which destroyed several hours work of Mikea children’s tuber digging. Finally, \cite{crittenden_juvenile_2013} further  demonstrated that by focusing on fruit, shallow tubers, small mammals, and birds, some Hadza children could produce over 100\% of their daily caloric needs. 

These studies suggest that children’s foraging decisions reflect different trade-offs than those of adults. However, little research has examined how the resources children target change as they gain skill and strength with age.

\subsection{Cooperative Breeding}
Even among the Hadza, arguably the most active child foragers, children do not regularly forage enough to feed themselves. Instead, Hadza children, like all human juveniles, rely on the support of a wide array of alloparents, who provide care and food (\cite{crittenden_allomaternal_2008}). This is because humans are cooperative breeders, i.e. group members other than the parents help care for and provision offspring (\cite{hrdy_evolutionary_2006}). The presence of multiple caretakers is considered to have had important consequences on the evolution of life history in our species. 
For mothers, offsetting at least some of the cost of childcare has resulted in shorter inter-birth intervals than other great apes (\cite{ meehan_cooperative_2013}). For offspring, not being completely responsible for self provisioning after weaning allows more flexibility in allocation of both time and resources.
%For grandmothers…For children, having access to multiple caretakers may have resulted in the lengthening of childhood, because…  \textit{ This sentence isn’t clear enough—can you rewrite it to get at the evo logic behind alloparents=longer childhood?
%“Moreover, once the presence of a dependant offspring is not preventing further investment in fitness, the same offspring can remain much further in the dependencies of caretakers -allowing space for the evolution of childhood- without hindering too much their parent’s reproduction .”}
Research on alloparents has primarily focused on provisioning from fathers and grandmothers (\cite{hawkes_hadza_1997, gibson_helpful_2005, sear_effects_2002}). Much less is known regarding how children themselves help offset their own, and their siblings’, cost of care (\cite{reiches_pooled_2009, kramer_early_2009, kramer_pooled_2010}). And yet, their contributions may be substantial (\cite{cain_economic_1977}).
%in as  much as the activities of individual household members are coordinated  parts of a single household enterprise\textit{ Need specifics—what does Cain FIND?}. 
For example, \cite{stieglitz_household_2013} showed that Bolivian Tsimane children’s labour can substitute that of absent fathers. Among Mexican Maya, children’s contributions to the household economy through domestic chores and childcare funds “between 82 per cent and 93 per cent of their own costs and between 35 and 52 per cent of the family’s total consumption during the mid-family cycle” (\cite{kramer_variation_2002, kramer_maya_2005, kramer_childrens_2005, kramer_does_2009}). These accounts, however, mostly focus on children’s time allocation to work activities and thus, do not provide detailed accounts of direct nutritional provisioning from juveniles.






%For example, the appearance of a post reproductive period in life seems to have been selected because it allows older individuals, especially women, to care for their grandchildren and hence increase their fitness (\cite{hawkes_2005_human}). 
%Most importantly, lifting at least part of the weight of childcare from the shoulders of mothers, gives them the possibility to have a new pregnancy earlier, shortening the average inter-birth interval. Moreover, once the presence of a dependant offspring is not preventing further investment in fitness, the same offspring can remain much further in the dependencies of caretakers -allowing space for the evolution of childhood- without hindering too much their parent's reproduction. This creates the peculiarly human case of families with multiple dependent offspring as well as variably related caretakers, all of which share resources in some sort of pooled energy budget (\cite{reiches_pooled_2009, kramer_early_2009, kramer_pooled_2010}). It is well clear that the different members contributing to this shared pool might have different stakes at play, and much research has been dedicated to the different engagement of mothers, fathers as well as other stakeholders such as grandmothers. Much less is understood of how children themselves are implicated in this exchange, and their role as producers is often overlooked. As relevant examples, \cite{cain_economic_1977} points out the importance of children's contribution to the household's workload, or \cite{stieglitz_household_2013} shows that children are more likely to engage in tasks related to food in the absence of their fathers. In general, it has been demonstrated that children’s contributions to the household economy through domestic chores and childcare offsets a significant portion of their own, and their siblings’ cost of care, for example in the work among the Maya of \cite{kramer_variation_2002, kramer_maya_2005, kramer_childrens_2005, kramer_does_2009}. These accounts, however, mostly refer to time allocation data and do not provide detailed accounts of the proportion of food contributed. 

\subsection{Embodied Capital Theory}
Another aspect of human life history invoked in studies of children’s foraging is the length of the pre-reproductive period itself. The Embodied Capital Theory (ECT) proposed by \cite{kaplan_theory_2000} has focused on understanding the co-evolution of human longevity, large brains, and our skill-intensive foraging niche. Considering these factors together, proponents of ECT argue that childhood evolved as a period for learning, during which children acquire skills and knowledge  that will allow them, as adults, to successfully extract nutrient-dense resources from their environment (\cite{kaplan_theory_2000, kaplan_embodied_2001, kaplan_embodied_2003, kaplan_neural_2003, kaplan_evolution_2007, kaplan_theory_1996, kaplan_evolution_1997, kaplan_life_2006, kaplan_emergence_2002}). Research testing this hypothesis tends to focus on how experience correlated with foraging returns, independently of strength or size. According to ECT, return rates from foraging and daily amount of food collected should increase after foragers acquire the skills and knowledge to extract these resources. 
%("Our view is that human childhood is elongated by including a period of very slow physical growth, during which the brain is growing, learning is rapid, and little work is done. This is followed by adolescence, during which growth is accelerated so that the brain and body can function together in the food quest. Early adulthood is a time for vigorous work during which resource acquisition rates increase through on-the-job training.")

Support for ECT largely comes from research on adolescents and adult males engaging in a hard task such as hunting. Ache, Tsimane, and Gidra men’s hunting efficiency peaked in mid-adulthood, some ten years after peak physical strength, suggesting that experience plays a central role in the development of hunting skill (\cite{walker_age-dependency_2002, gurven_how_2006, ohtsuka_hunting_1989}). \cite{koster_life_2020} conducted the first cross-cultural study of men’s hunting returns, with data from over 1,800 individuals from over 40 societies. Their findings largely support ECT, with age of peak hunting success averaging between 30 and 25 years of age. However, the authors note considerable individual and cross-cultural variation in hunting returns, likely due to environment, prey type, prey encounters, and individual motivation. 

Studies focusing on all foraging returns rather than solely hunting, and on younger children rather than adolescents and adults, have provided more mixed support for ECT.

\cite{bock_learning_2002} found that competence, acquired through both experience and physical growth, is reached earlier for simpler tasks compared to more complex ones, suggesting support for ECT. Other studies find limited evidence for the importance of skill acquisition of children’s foraging success. 
For example, \cite{ bird_constraints_2002, bird_children_2002, bird_mardu_2005} argue that Mardu and Meriam children’s foraging returns are primarily restricted by their small size.
\cite{blurton_jones_selection_2002} further found that Hadza children who had attended boarding schools, and thus, had little bush experience, did not underperform in tubber-digging compared to bush-residing counterparts. Finally, while \cite{crittenden_juvenile_2013} found that Hadza foraging returns increased with age, young children can collect enough food to exceed their daily caloric needs if needed, suggesting that motivation rather than age influences children’s returns. Taken together, these studies point to the effects of growth and skill acquisition on children’s foraging success will depend on the types of resources available in their ecological niche, which requires cross-cultural investigation. 
%\textit{I think we need a stronger closing here…}

\subsection{The Present Study}
In summary, while there is a long history of studying children’s foraging returns, ecologically-grounded and empirical cross-cultural work on children’s direct nutritional provisioning is still needed to answer outstanding questions regarding children’s optimal foraging, alloparenting, and skill acquisition. In their study of hunting returns, \cite{koster_life_2020} demonstrated the strength of a cross-cultural comparative approach to understanding the development of human foraging, and its implications for resolving current human life history debates. Here, we expand beyond hunting by presenting the first meta-analysis of child and adolescent foraging returns. Our study has two main goals. First, we examine age-dependent variation in children’s foraging returns, in order to elucidate the factors that may constrain and enable children’s foraging participation across cultures. Second, we assess how methodological and analytical differences introduce heterogeneity in studies of children’s foraging returns, in order to promote the comparability of future studies.  





\subsubsection{demand and family budget + payoff for work}
how San and Hadza children and their parents coordinated their labour to maximize foraging returns (\cite{hawkes_hadza_1995}). For example,  These early studies were the first to show that children’s foraging should be considered in relation to their ecological and family context, and that children could make substantial contributions to foraging returns, even if they would not always do so.
%Demand (captured perhaps by number of dependents in the family, producer consumer ratios, etc), linked into coop breeding literature, that there has been strong kin selection for helping with the provisioning of close relatives. Here I am not clear how there is any clear prediction about how this might be associated with age, other than that foraging may start at younger ages in populations where there is more demand for young children to act as additional provisioners, perhaps because of smaller camp sizes, greater risk of food shortages? (I really don’t know).

\cite{bird_constraints_2002}
 %Adult foraging strategies are bound up with adult social and reproductive strategies, creating payoffs for certain prey choice and time allocation decisions that may not result in energy maximization (e.g., Bliege Bird et al. 2001; Hurtado et al. 1992). Because children do not yet face these opportunities and trade-offs, even if they learn quickly and are strong enough, they may still not forage like adults.
\cite{blurton_jones_modelling_1989}
 %effect of children foraging on interbirth interval of mothers - cause other way around? children can forage hence mothers can have more children?
\cite{bock_evolutionary_2002}
 %The availability of laborers and the overall labor requirements of the household are major determinants of investment in alternate forms of embodied capital and resulting variation in children's time allocation. The value of children's labor to their parents is dependent upon the opportunity costs to engaging in other activities not only for the child in question but also for potential substitute laborers.
\cite{bird_constraints_2002}
 %The Meriam data show that Meriam children learn quickly how to forage efficiently given their size constraints, and that increases in efficiency across the lifespan could be due to accumulated experience, but because we do not see gradual cumulative increases, it may be more likely that these increases in efficiency are due to increases in the benefits of working harder.
 



Differences between !Kung and Hadza (\cite{blurton_jones_differences_1994})
\cite{blurton_jones_foraging_1994}
%Childreno f theh untinga ndg athering!K ungS an seldomfo raged,e speciallyd uringt he longd rys eason.I n contrastc, hildreno fH adzaforagersin Tanzaniao ftenf orage,i n both wet and drys easons

\cite{bird_children_1995}
 %It shows that the activities of children can be productive and valuable in particular environments, and that children, from a very young age, are capable of contributing much to their own and their families' subsistence. Such support may have a profound effect on parental activity and work patterns.

\subsubsection{danger of the environment}
%Danger of the environment (linked to early comparison of hadza and kung stressing costs of foraging). One might expect, from the inferences drawn from that comparative study, that kids will invest less in foraging at young ages in dangerous than less dangerous environments.

\cite{blurton_jones_differences_1994}
% !Kung children appear at considerable rish of getting lost if they wander in the bush. Furthermore data elsewhere show that because of the way the food is distributed in the environment,!Kung children cannot acquire much food withouth travelling far from dry season camps that are near permanent water. Hadza children have many landmarks by which to navigate and acquire much food close to camp.



\subsubsection{knowledge and skills}
%Knowledge & skill (e.g., studies like yours and Jeremy’s that directly measure knowledge or Nick’s that measured actual skill in hadza olympics) linked into ECT (specifically skill-intensivity).  Insofar as ECT, in proposing an explanation for our long childhood, draws attention to the skill intensivity of our foraging niche, we would predict that there will be ecological predictors of how skills increase with age – slower in skill-intensive environments and skill-intensive tasks than in non skill-intensive environments and tasks, each of which you would expect to pick up with foraging returns.
\cite{bird_children_2002}
 %How much experience do Meriam children need before they become efficient reef foragers? Evidently very little (follow more discussion)
\cite{bird_mardu_2005}
 %Only when walking speeds approach the adult average does hunting in sandhills consistently offer higher efficiency. These data may be consistent with the argument that prolonged human juvenility evolved for reasons other than to learn complex hunting strategies,
\cite{blurton_jones_why_1997}
 %we are probably wrong to thuink that childhood evolved in order for more learning to occur. (if children would forage only to learn, the !Kung, whose chidlren do not forage, would have very little to learn - not super sound logic)
\cite{blurton_jones_selection_2002}
 %These findings do not support what we label "the practice theory,"
 %They differ greatly in time spent practicing digging but show no difference in efficiency. Children who lost "bush experience" by spending years in boarding school performed no worse at digging tubers or target archery than those who had spent their entire lives in the bush. Climbing baobab trees, an important and dangerous skill, showed no change with age among those who attempted it. We could show no effects of practice time.
\cite{bock_evolutionary_2002}
 %The development of adult competency in specific tasks entails a steplike relationship between growth- and experience-based forms of embodied capital in the ontogeny of ability acquisition.
 %There is a trade-off between the acquisition of experience-based embodied capital in the form of skills and knowledge and immediate productivity among children. Time allocation to these alternatives is primarily determined by the short- and long-term costs and benefits to parents of investment in children's embodied capital. 
 \cite{bock_subsistence_2004}
 %role of play for learning subsistance skills
\cite{bird_constraints_2002}
 %However, for shellfish collecting, which is relatively easy to learn, there are strong age-related effects on efficiency. Children reach adult efficiency more quickly in fishing as compared to shellfish collecting, probably owing to the size and strength constraints of the latter.





\subsubsection{anthropometrics}
%Variables captured by anthropometric measures (Strength/speed/height/etc) linked I think to Bogin’s theoretical perspective, and evidenced in Bock, Birds, etc work. etc. Here the predictions would be foraging returns reflect the age-specific increases in each of these physical traits.



%Then point out that neither of these make specific predictions about how exactly kids’ foraging success should vary with age (which we know it is highly variable across cultures, even foraging cultures). Suggest that therefore we need a systematic understanding about how the age-specific changes in foraging efficiency are affected by the affordances of the ecology, demography, and subsistence alternatives. 



%The changes happening in the phenotype of children and teenagers imply that, on the one hand, we can use the predictions from OFT to study foraging behavior in young individuals. But also, on the other hand, that measuring what children and teenagers target when foraging and how their returns change with age, can give new insight on the factors influencing foraging behavior in general.


Understanding this variation, especially in hunter gatherer populations, has been the goal of much research since the '90s (e.g. \cite{blurton_jones_modelling_1989, blurton_jones_foraging_1994, blurton_jones_why_1997, bird_children_1995}). %find better hook



Another relevant aspect that determines prospected returns is knowledge of the environment and skills each individual possesses. These can be acquired through training, potentially at the cost of short term benefits. Humans exploit complex foraging niches that can yield high returns, but require individual competencies, sometimes defined embodied capital from economics (\cite{kaplan_theory_2000}). It has been hypothesized that human life history itself has been selected so that individuals have time to acquire these competencies during a long pre-reproductive period during which they are at least partially dependent on the support of adults. Young individuals, then, can be faced with the additional choice between maximizing current returns by aiming at an easier target or focus on a harder prey with lower probability of success, while acquiring skills and improving future returns by doing so. 

but this doesn't hinder adult reproduction as much as in our close relatives, because mothers become fertile again much before children become independent. This implies that families are composed by multiple dependent children at the same time, who 



\cite{bird_ethnoarchaeology_2000} 
 %Such opportunity costs of foraging are particularly relevant to archaeological problems when the goal of foraging is to maximize delivery rate of a resource rather than the rate of acquisition per se
\cite{bird_children_2002}
 %This difference and the precise nature of children's selectivity while reeffiat collecting are consistent with a hypothesis that both children and adults attempt to maximize their rate of return while foraging, but in so doing they face different constraints relative to differences in walking speeds while searching.
\cite{bird_mardu_2005}
 %Moreover, children’s decisions to hunt in rocky outcrops as opposed to the sandhills (that adults target) are not likely to be the result of learning constraints. By focusing their efforts in rocky outcrops, children (who walk slower than adults) can encounter prey at a higher rate. On average, this provides return rates for children that are equivalent to those they might expect if they hunted in the sandhills, while avoiding the long search distances involved in sandhill hunting.
\cite{blurton_jones_foraging_1994}
%trategyW. ec alculatet he benefittso a !Kungm otherif hero ldestc hilda ccompaniehde r to then ut groves.B ecauseo f theh ighp rocessingc osts, a child'sw orkt imew as mostp rofitablsyp enta t homec rackingn uts

\subsection{evolution of childhood and ECT}
%and the perspective of ECT (to target your paper towards the puzzle of the evolution of childhood). 
\cite{blurton_jones_why_1997} 
%children would have been selected to make trade offs between current survival, promotion of kin and future reproductive prospects
\cite{blurton_jones_selection_2002}
%Humans have a much longer juvenile period (weaning to first reproduction, 14 or more years) than their closest relatives (chimpanzees, 8 years). Three explanations are prominent in the literature. (a) Humans need the extra time to learn their complex subsistence techniques. (b) Among mammals, since length of the juvenile period bears a constant relationship to adult lifespan, the human juvenile period is just as expected. We therefore only need to explain the elongated adult lifespan, which can be explainedby the opportunity for older individuals to increase their fitness by providing for grandchildren. (c) The recent model by Kaplan and colleagues suggests that longevity and investment in "embodied capital" will coevolve, and that the need to learn subsistence technology contributed to selection for our extended lifespan.

Effect of foraging activities of children on life history: interbirth interval (\cite{blurton_jones_modelling_1989})

\subsection{age dependency of foraging returns and its variation}
%Then point out that neither of these make specific predictions about how exactly kids’ foraging success should vary with age (which we know it is highly variable across cultures, even foraging cultures). Suggest that therefore we need a systematic understanding about how the age-specific changes in foraging efficiency are affected by the affordances of the ecology, demography, and subsistence alternatives. 
%Then turn to the evidence for each of the variables hypothesized/shown to affect kids foraging choices and returns, and (this is difficult) make predictions (or at least empirically derived expectations) for how these might interact with age across different contexts: 
%I appreciate that some of these hyps are tested more WITHIN populations, and some between populations, but this can easily be done with a multilevel model.

Differences between !Kung and Hadza (\cite{blurton_jones_differences_1994})
\cite{blurton_jones_foraging_1994}
%Childreno f theh untinga ndg athering!K ungS an seldomfo raged,e speciallyd uringt he longd rys eason.I n contrastc, hildreno fH adzaforagersin Tanzaniao ftenf orage,i n both wet and drys easons

\cite{bird_children_1995}
 %It shows that the activities of children can be productive and valuable in particular environments, and that children, from a very young age, are capable of contributing much to their own and their families' subsistence. Such support may have a profound effect on parental activity and work patterns.

\subsubsection{danger of the environment}
%Danger of the environment (linked to early comparison of hadza and kung stressing costs of foraging). One might expect, from the inferences drawn from that comparative study, that kids will invest less in foraging at young ages in dangerous than less dangerous environments.

\cite{blurton_jones_differences_1994}
% !Kung children appear at considerable rish of getting lost if they wander in the bush. Furthermore data elsewhere show that because of the way the food is distributed in the environment,!Kung children cannot acquire much food withouth travelling far from dry season camps that are near permanent water. Hadza children have many landmarks by which to navigate and acquire much food close to camp.




\subsubsection{demand and family budget + payoff for work}
how San and Hadza children and their parents coordinated their labour to maximize foraging returns (\cite{hawkes_hadza_1995}). For example, San mother-child dyads produced more kcal/hour  when children remained in camp to process \textit{mongongo} nuts than if they helped their mothers collect them. Hadza children, on the other hand, routinely accompanied their mothers to predictable berry patches, which maximized mother-child total yield. These early studies were the first to show that children’s foraging should be considered in relation to their ecological and family context, and that children could make substantial contributions to foraging returns, even if they would not always do so.
%Demand (captured perhaps by number of dependents in the family, producer consumer ratios, etc), linked into coop breeding literature, that there has been strong kinselection for helping with the provisioning of close relatives. Here I am not clear how there is any clear prediction about how this might be associated with age, other than that foraging may start at younger ages in populations where there is more demand for young children to act as additional provisioners, perhaps because of smaller camp sizes, greater risk of food shortages? (I really don’t know).

\cite{bird_constraints_2002}
 %Adult foraging strategies are bound up with adult social and reproductive strategies, creating payoffs for certain prey choice and time allocation decisions that may not result in energy maximization (e.g., Bliege Bird et al. 2001; Hurtado et al. 1992). Because children do not yet face these opportunities and trade-offs, even if they learn quickly and are strong enough, they may still not forage like adults.
\cite{blurton_jones_modelling_1989}
 %effect of children foraging on interbirth interval of mothers - cause other way around? children can forage hence mothers can have more children?
\cite{bock_evolutionary_2002}
 %The availability of laborers and the overall labor requirements of the household are major determinants of investment in alternate forms of embodied capital and resulting variation in children's time allocation. The value of children's labor to their parents is dependent upon the opportunity costs to engaging in other activities not only for the child in question but also for potential substitute laborers.
\cite{bird_constraints_2002}
 %The Meriam data show that Meriam children learn quickly how to forage efficiently given their size constraints, and that increases in efficiency across the lifespan could be due to accumulated experience, but because we do not see gradual cumulative increases, it may be more likely that these increases in efficiency are due to increases in the benefits of working harder.
 
 
\subsubsection{time budget and competing activities}
%Time budgets (arising from allocation to other skills acquisition like formal education) which would be linked basically into the opportunity costs associated with foraging (and therefore OFT writ broad (why go on foraging when there are better returns to your time). Here again the predicted associations with age is a bit problematic, but if there are alternatives like schooling available, you might predict a reduction in the rate at which childrens’ returns improve with age after they enter school).




the type and amount of foraging done by young individuals differs substantially , and variation could arise from different components.
Most obviously, kids of different ages often do different things(\cite{}), so that a first goal would be to clearly define how foraging returns vary as individuals grow older. Age, though, represents a proxy for other factors that change with passing time, such as body size, strength, or, as proposed by the ECT, knowledge. Moreover, age-specific changes in foraging efficiency are affected by ecology, demography, and subsistence alternatives, among other factors. 




















Children and teenagers all over the world roam their environment to collect food. This behavior can vary in intensity from casual berry picking, common in northern Europe during summer, to a regular activity providing a consistent proportion of the individual's diet, as happens among Hadza children (some of which can collect more than 1000 kcal a day by the age of 10, see \cite{crittenden_juvenile_2013}). 



Despite being widely prevalent, foraging behavior has been studied only sparsely in children in the last few decades (starting in the '90s with examples such as \cite{blurton_jones_modelling_1989, blurton_jones_foraging_1994, blurton_jones_why_1997, bird_children_1995}).

Foraging behavior of children have been suggested to bear implications for the evolution of human life history, in particular of childhood itself. Human offspring remain dependant from adults for much longer than other ape juveniles, and are provisioned by parents and alloparents while they start producing their own food (\cite{hrdy_evolutionary_2006, crittenden_allomaternal_2008}).%check better reference for -time until autonomy + coop breeding
According to the Embodied Capital Theory (ECT) proposed by \cite{kaplan_theory_2000}, this period of time between the start of provisioning and independence provides a buffering zone during which individuals can acquire knowledge and skills which will provide higher benefit during adulthood (see also \cite{kaplan_embodied_2001, kaplan_embodied_2003, kaplan_neural_2003, kaplan_evolution_2007, kaplan_theory_1996, kaplan_evolution_1997, kaplan_life_2006, kaplan_emergence_2002}). Children would then use foraging strategies which do not yield maximum immediate returns, but rather improve the ability to exploit the complex foraging niche humans occupy. 

Optimal Foraging Theory (OFT) has commonly been the theoretical background used to frame this questions (\cite{winterhalder_hunter-gatherer_1981, smith_anthropological_1983}). OFT relies on the assumption that foraging behavior has been selected so that individuals make choices that maximize their fitness - survival or reproductive success (\cite{pyke_optimal_1977}). When foraging, individuals are faced with a set of constraints, and can decide between some available options, or different behaviors, while attempting to maximize returns of their effort. Given the difficulty in measuring fitness in humans, different currencies are commonly employed in analyses as proxies, for example the rate of returns per unit of foraging time (\cite{smith_anthropological_1983}).

So, for example, a forager who sets off to procure the day's meal is limited by the availability of resources in the environment or current season, or by other constraints such as cultural food taboos. By deciding which resource to target among the available options, this individual is also deciding whether to maximize net energy return per unit of time, protein intake, or even social capital, for example aiming at a prestigious kill (\cite{hawkes_foraging_1996}). The goal of OFT has often been to infer the goal of foragers, and hence the motivation behind the behaviors, from the choices they make when foraging. 

Using a OFT framework, it should be possible to make clear predictions on how children should behave when foraging and interpret this behavior to improve out understanding of ECT and other hypotheses. Given the variation present in and across different societies(\cite{koster_life_2020, }), %cite more on the variation between societies
though, this task is not very straightforward. In this paper we will first discuss why should we expect foraging behavior to change with age, and which other factors are expected to influence its variation. We will then tackle the first step on the way to understanding children foraging behavior, by modeling how returns vary with age. To do so, we conduct a meta-analysis of data available in the literature with a bayesian bla bla model. We also consider variation due to gender differences and to different targeted resources. We conclude by discussing how our results improve our understanding of children foraging behavior and the implications for broader hypotheses for the evolution of human life history. 

\subsection{Age dependency of foraging returns and its variation}
Foraging behavior and the returns children obtain are expected to vary as they grow older, since the payoffs they can expect from investing in certain activities change. Increase in returns with age has been measured by \cite{bird_constraints_2002, crittenden_juvenile_2013}. %add citations on increase with age or just say in the literature
It is not clear, though, how this change takes place nor how it can differ according to its contingencies. Age itself represents a proxy for other factors that change with passing time, such as body size, strength, or, as proposed by the ECT, knowledge (\cite{bird_children_2002}). Moreover, age-specific changes in foraging efficiency are affected by ecology, demography, and subsistence alternatives, among other factors. 


Most obviously, a growing body and increasing strength imply that most tasks should have age specific return rates (\cite{koster_life_2020, crittenden_juvenile_2013}). Physical traits that change in association to age have been described as 'growth based embodied capital' (\cite{bock_evolutionary_2002}), but the specific effect of each of these components has not been tested. 
More support is given to the hypothesis that children target different resources as they grow in accordance to what gives higher yields for their body size. For example, Rebecca and Douglas Bird \cite{bird_ethnoarchaeology_2000, bird_constraints_2002} argued that because Meriam children choose to exploit a wider array of prey types because they walk slower and encounter high-ranked resources at a lower rate than do adults. Moreover, in \cite{bird_mardu_2005}, they demonstrate that, given that larger body size reduces the cost of walking for Mardu children in the Australian desert, individuals migrate from targeting children specific to adult specific prey types as they grow, because it becomes profitable to reach certain patches of resources rather than others. \cite{tucker_growing_2005} among Malagasy Mikea showed that children focused on collecting younger and shallower tubers in patches abandoned by adults. Finally, \cite{crittenden_juvenile_2013} further  demonstrated that by focusing on fruit, shallow tubers, small mammals, and birds, some Hadza children could produce over 100\% of their daily caloric needs. 
We hence expect an increase in returns with age simply due to growth, as an effect of increasing speed, body size, strength or even muscular coordination. We also expect to observe different trajectories for different tasks and variation due to the cultural/ecological environment. Finally, we expect children to target different resources as they grow. 

 
But an adult body is sometimes not enough to achieve maximum return rates from foraging. Ache, Tsimane, and Gidra men’s hunting efficiency peaked in mid-adulthood, some ten years after peak physical strength (\cite{walker_age-dependency_2002, gurven_how_2006, ohtsuka_hunting_1989}). As suggested by the ECT,  knowledge of the environment and other traits acquired through experience probably play a central role in the development of hunting skill. These traits have been defined 'practice based embodied capital' and can be acquired through training, which comes potentially at the cost of immediate returns (\cite{bock_evolutionary_2002}). The investment in training would be repaid by the benefits of exploiting hard to get resources that can yield high returns, but require complex sets of competencies. Such an hypothesis is difficult to test, but a range of predictions have been made on the effect of practice on returns. These include the fact that we should expect variation in age at peak of returns for different environments or resource types, depending on the skill sensitivity of the task. For example, \cite{bock_learning_2002} found that competence, acquired through both experience and physical growth, is reached earlier for simpler tasks compared to more complex ones. Moreover, \cite{koster_life_2020} conducted the first cross-cultural study of men’s hunting returns, with data from over 1,800 individuals from over 40 societies. Their findings largely support ECT, with age of peak hunting success averaging between 30 and 25 years of age. However, the authors note considerable individual and cross-cultural variation in hunting returns, likely due to environment, prey type, prey encounters, and individual motivation.
\cite{bird_constraints_2002} seem to find opposing evidence from their comparison of how foraging efficiency increases across different foraging types performed by Meriam children and adults. Indeed, they find strong age-related effects for shellfish collecting, which they consider relatively easy to learn; on the contrary children reach adult efficiency more quickly in fishing which they consider harder. They explain this pattern with size and strength constraints.
Another prediction is that individuals deprived of opportunities for learning should have a lower return rate. \cite{blurton_jones_selection_2002} compared the return rates associated to digging tubers in Hadza of different ages who differed in whether they attended or not boarding school off camp. The expectation that individuals who spend long periods of time in school with no opportunity to practice tuber-digging skills should perform better was not met. No difference was observed between sexes either, despite boys stop digging tubers when still young, whereas girls spend long time in this activity - with no apparent benefit to productivity. 

The relation of knowledge itself with age has been tested only sporadically, mostly not in correlation with foraging returns.%cite papers on knowledge - maybe
\cite{koster_wisdom_2016} observes an increase in knowledge with age, and discusses the implications this has for efficiency, but does not associate data on knowledge with foraging returns.

Despite most authors do not seem to support the idea that large amounts of knowledge are necessary to be a proficient hunter (\cite{blurton_jones_why_1997, bird_children_2002, bird_mardu_2005}), definitive conclusions are hard to draw, as the effect of learning are not easily quantified. 
For example could include learning through play (\cite{bock_subsistence_2004}).%check for sheina's paper and others on learning - maybe
\cite{bock_evolutionary_2002} hypothesizes a step-like relation between growth and experience based forms of embodied capital, in which productivity in certain tasks can be achieved only with a combination of skills and strength. 
Most relevantly, though, knowledge and skills would provide different benefits for different tasks, depending on the complexity. \cite{koster_life_2020} argues that hunting efficiency peaks particularly late in life probably because it requires a vast array of both physical and knowledge based abilities -for finding preys, tracking, aiming, killing or transporting-, whereas efficiency in easier activities such as berry picking should not increase substantially after a certain level of ability is reached. 
%Lack of support for this hypothesis could be due to small (need lots of trips to define quality of foragers) or incomplete(need both estimation of knowledge, of returns, of skills) data or the necessity of complex stats to pick up the patterns 
Measure how returns improve with age in different tasks and societies would be an important step to shed light on this questions.


As mentioned, the natural environment provides a source of variation for returns across ages and societies. The first studies looking at children foraging behavior focused on the characteristics of the environment to explain the observed differences between San and Hadza children. Specifically, differences in water availability, risk of predation and the proximity of resources to settlements appear to explain why San children collected almost none of their own food resources, while Hadza children actively contributed to their own subsistence \cite{blurton_jones_differences_1994}. %check for more in the literature

What children do and how much they forage depends also on the social environment, i.e. the structure of the household and associated division of labor (\cite{bock_evolutionary_2002}). The uniquely human trait of resuming reproduction before the previous offspring is completely independent implies that reproducing individuals have often multiple dependent offspring at the same time. This can be sustained, among other things, through sharing of resources and integration of tasks. Children participate actively to the household economy and provide valuable support (\cite{cain_economic_1977}). Among Mexican Maya, for example, children’s contributions to the household economy through domestic chores and childcare funds “between 82 per cent and 93 per cent of their own costs and between 35 and 52 per cent of the family’s total consumption during the mid-family cycle” (\cite{kramer_variation_2002, kramer_maya_2005, kramer_childrens_2005, kramer_does_2009}). Hence, the intensity at which they forage and the targets they choose is influenced by the the overall labor requirements of the household (\cite{bock_evolutionary_2002}). For example, San and Hadza children and their parents have been observed to coordinate their labour to maximize foraging returns (\cite{hawkes_hadza_1995}). San mother-child dyads produced more kcal/hour when children remained in camp to process \textit{mongongo} nuts than if they helped their mothers collect them. Hadza children, on the other hand, routinely accompanied their mothers to predictable berry patches, which maximized mother-child total yield.

It is hard to explicitly predict how household composition influences children foraging behavior, also because the productivity of children itself has an effect on the structure of the household, because when children can contribute more substantially to their maintenance, women can have shorter inter-birth intervals and higher fertility (\cite{blurton_jones_modelling_1989}). It has been recorded, though, that social situations has effect on how much children forage. \cite{stieglitz_household_2013} showed that Bolivian Tsimane children’s labour can substitute that of absent fathers. Anecdotally, \cite{crittenden_juvenile_2013} noted that two outliers in her sample of returns for age corresponded to two girls who lost their parents and hence were more motivated to invest more in their maintenance, and more generally, the fact that young children can collect enough food to exceed their daily caloric needs if needed, but don't always do so, suggests that motivation greatly influences children’s returns. 

Finally, the motivation that pushes individual to forage can change with age, implying a change in the targeted resources and returns. Signaling quality as a hunter with prestigious kills has been adduced as a reason for why males engage in hunting activities that have uncertain results and less than average outcomes (\cite{hawkess?}), and this motivation could appear only at some point during teenager years, and not during childhood. But more in general children do not face the same trade-offs as adults do, and as social and reproductive goals become relevant, they are expected to modify their foraging strategies to maximize different outcomes (\cite{bird_constraints_2002}). 

The sample we constructed from data available in the literature does not allow to explore all of these sources of variation at the same time. While we call for more diverse and complete data collection of foraging returns in children, (possibly including data from non hunter gatherer societies, see \cite{lee_child_2007,lee_childrens_2009} for some accounts of children foraging in a Mexican shanty town), we focus our analysis on quantifying the change in foraging returns with age, taking into account sex differences and an effect of foraged resources.


















To understand how age specific returns are influenced by these differences among societies and resources, we model how foraging returns increase with age with a fully bayesian model. 

By focusing on age specific returns, we managed to account for the difference


We do so by modelling how foraging returns increase with age with a bla bla bayesian model. We also account for differences among sexes and among different resources targeted (e.g. tubers, game, etc).

In order to bring more clarity on how children foraging influences life history, we model how foraging returns increase with age using data sets available in the literature. We account for differences among sexes and among different resources targeted (e.g. tubers, game, etc). This is only a first step 

Our approach is not sufficient, as all the variables involved should be available for analysis (physical characteristics, knowledge, 

The importance of knowledge acquisition through foraging, and the effect th
The relation between children foraging and life history, as mediated by the acquisition of knowledge, is a harder to tackle subject. 
It has been suggested that the long prp in humans exists because we need to learn about the environment. If this is so, foraging would have a pivotal role in ability acquisition. 



The consequences of children self provisioning have been demonstrated already. The association between child foraging and ability acquisition on the other hand 

The causal mechanism by which children ability acquisition through foraging has an effect on the evolution on life history traits is not as clear. 


The need to acquire abilities through foraging has been suggested as a reason for which the prp elongated in humans, but the causal mechanism is not clear. 


The consequences of children self provisioning have been demonstrated already. In the San of the Kalahari desert the interbirth interval has been shown to be much larger then the one among the Hadza. This comes as a consequence of children foraging behavior, as argued by \cite{blurton_jones_modelling_1989}. It is obvious that with longer interbirth interval we have a reduction in fitness. 


When roaming the wild, children provide food for themselves, but also acquire 


