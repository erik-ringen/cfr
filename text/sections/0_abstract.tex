\section{Abstract}
Children forage in a vast array of societies, spanning from traditional hunter gatherer groups such as the Hadza to mexican shanty towns. Data on their success, though, is sparse and a coherent framework is missing. With this paper we aim at both reviewing the extant literature on children foraging and at summarizing available return data in a single coherent meta-analytical framework. Moreover we focus on one specific hypothesis tested by some of the papers we analyse, and discuss how these papers draw their conclusions from data. 





%meta-analysis we aim to summarize previous research both qualitatively and quantitatively. Starting from a cross culturl analysis published by \cite{koster_life_2019}, we developed a model to analyze published data on children foraging returns. 

%Children foraging activities interest different areas of the study of human behavioral ecology. The hunting and gathering of children and adolescents has a role in household economy and division of labor studies, cooperative breeding in humans and in investigating the evolution of life history traits. It is surprising, then, that data is scarce and a unifying framework is missing. In this paper, we compare the available data on children foraging returns in the current literature and argue that: - children forage extensively in several societies, - differences in productivity can be explained by ecological contingencies, - the improvements with age vary across culture and task. We also address the context in which research on children foraging has been conducted and call for an increase in data collection as well as for sharing protocols and methods. A concerted effort to create datasets comparable across cultures would allow to address important questions of human evolution.


