
\cite{blurton_jones_selection_2002}
%"Our data also show that it is not safe to assume that increases in skill with age are entirely due to learning or practice; they may instead be due to increases in size and strength."
%We experimented on three important subsistence skills. We found that one, using pegs to climb baobab trees, reached adult proficiency late but rapidly. The other skills, digging tubers, and target shooting, showed no effects of large differences in opportunity for practice. We showed statistical effects of age, size, and strength on these essential skills but no convincing effect of practice. The ten or more years of daily digging that girls have experienced by their teens left them producing tubers at no greater rate than boys, with only five years of experience. One to six years of boarding school (where boys are not allowed to keep bows) left them no worse at hitting the target than boys who had avoided boarding school and spent their entire lives in the bush, daily roaming with bow and arrows. It would be unwise to translate age differences as unambiguous evidence of practice effects. Some practice must be essential for these skills. But if we are to use the value of practice to explain why humans mature at 18 while chimpanzees mature at 13, it would seem that we should be able to show effects of differences in practice on this kind of time scale. We do not see them. These results offer no comfort to the view that humans mature late because they have so much to learn. [...]
%The most important limitation in our study is that we have so far only attended to three skills. Life as a hunter, or as a gatherer, employs many more skills. For the hunter, tracking and stalking are obviously important, and the extensive knowledge of animals and their behavior that men acquire (Blurton Jones and Konner 1976; McDowell 1984) may also be important. For women, not only would knowledge of the location and seasonality of many plant species be important but also skills involved in food processing (such as winnowing flour made from baobab fruit) and in child rearing. For both sexes social skills are involved in keeping the peace, developing and maintaining reputations, exchange, kin relationships, and marriages.
%We can think of several reasons why our experiments do not require anthropologists to immediately discard the practice theory. We will discuss (1) what we do not claim, (2) middle childhood versus teenage learning, (3) self-selection of the sample, (4) whether the deprivations were complete, (5) the power of the data, (6) the many additional component subsistence skills, (7) catch-up learning, (8) the juvenile time budget, (9) constraints on learning schedules, (10) social skills or cultural knowledge, and (11) evolutionary time scale.
%We do not claim there are no age differences in skill; we show that age is a strong predictor of digging and archery skill. We are concerned with whether these age differences can be better accounted for by learning and practice, or by growth and strength.
%Effects of lost practice in middle childhood might be much more striking. Our results do not contradict the suggestion of Bogin (1999) that middle childhood is a developmental period adapted for learning, and the teenage years a period for growing.
%Although our results appear to contradict several of Kaplan and colleagues' (2000) statements, our findings do not negate their formal model.
%MORE


\cite{bock_learning_2002}
%Among traditional populations in Liberia and Zambia, respectively, the anthropologists Lancy (1996) and Reynolds (1988) found that the development of skills related to adult productivity and social roles was extremely time intensive. Moreover, there is strong evidence that the acquisition and establishment of higher cognitive functions is slow, requires great effort, and can occur only through sustained formal or informal education (Ericsson et al. 1993; Geary 1995; Gelman 1993; Siegler 1996; Siegler and Crowley 1994)
%The results presented here provide strong support for the hypothesis that parental manipulation of children's time allocation to a suite of productive and non-productive activities is based on the short- and long-term costs and benefits to parents of alternative allocations. Moreover, the overall value of this time allocation is related to the basket of competencies relatedto adult capacity in a given time and place. The returns on investment in different forms of embodied capital are influenced by the value of returns on alternate forms of investment, the subsistence ecology, and the household demography and economy. These alternate investments constrain one another, but they also may work to punctuate development especially through the ontogenetic relationship of growth- to experience-based forms of embodied capital. 
%Embodied capital theory has major implications for understanding the complex polyinfluence of subsistence ecology, family and household demography and economy, and parenting behavior on children's time allocation to different activities. The returns on investment in two forms of embodied capital, growth-based and experience-based, vary as a function of not only these exogenous factors but also endogenous characteristics of children such as age and gender. 
%Four main findings emerged from this study: - The development of adult competency in specific tasks entails a steplike relationship between growth- and experience-based forms of embodied capital in the ontogeny of ability acquisition (the punctuated development model). - There is a trade-off between the acquisition of experience-based embodied capital in the form of skills and knowledge and immediate productivity among children. -The availability of laborers and the overall labor requirements of the household are major determinants of children's time allocation. The suite of competencies that relate to adult capacity varies across time and place.


\cite{walker_age-dependency_2002}
%Results support the argument that skill acquisition is an important aspect of the human foraging niche with hunting outcome variables reaching peaks surprisingly late in life, significantly after peaks in strength.
%The human foraging niche, and hunting in particular, is often interpreted as playing a central role in the evolution of human behavior and life history (e.g., Dart, 1953; White, 1959; Washburn & Lancaster, 1968; Isaac, 1978; Hill, 1982; Lancaster & Lancaster, 1983; Foley & Lee, 1991; Foley, 1992).
%This paper contrasts age schedules of Ache hunting ability with those of physical performance, models strength and skill effects on ability, and decomposes hunting ability into some of its constituent parts—e.g., finding game, killing game upon encounter, and archery ability
%We have found significant age effects on important measures of hunting ability (with body size controlled). Most notable of these are total number of prey found per hour and the probability of kill upon encounter with important prey species, especially monkeys. Peaks in these measures occur much later in life than age of peak physical performance.
%We could find no significant effect of body size on the probability of a successful pursuit. While strength is certainly an important factor in some foraging activities such as bow-and-arrow shooting, our results indicate that skill is more important in attaining proficiency in finding and killing prey. 
%Additionally, analysis of longitudinal notebook data from the inexperienced hunters shows no detectable improvement after 13·5 months. It remains to be seen if reservation-born Ache will ever perform at the same elevated level as their forest-born counterparts.
%Interesting comparisons can be made between our results and data from Olympic athletes as peak ages in ability in various sports appear to map on to the skill-intensity of the activity

\cite{bock_what_2005}

\cite{tucker_growing_2005}
%We argue that Mikea children are not trying to be “efficient” at all. There is little reason for them to be either rate-maximizers or time-minimizers. Because parents provision children from their surplus, children are not energy-limited. Because they have few alternative uses for their time, and their alternatives are not likely to bestow fitness advantages, they are also not time-limited. The life of a child in a small foraging camp is often quite dull. Children forage for the physical and mental challenge, and because it is an enjoyable social activity. During one focal follow, the senior author witnessed a “food fight” between the boys and the girls. Several kilograms of edible tubers were destroyed in the ensuing volley. For children, foraging is an extension of play that occurs outside camp.
%Consistent with their explanation, Mikea increase in foraging efficiency with increasing age, achieving their highest rates during adulthood. However, several aspects of age-specific foraging behavior among Mikea deviate from Kaplan et al.’s (2000) predictions. 
%First, Mikea children do not specialize on collected resources such as fruits and foliage, but rather, dedicate similar amounts of time to foraging for wild tubers, a high-quality “extracted” resource, as do older people.
%Second, Mikea children do not appear to be actively trained by older people. Mikea children experience the same return rates when foraging with potential trainers (adolescents and adults) as when foraging with other children. Women only forage with children during the early dry season, when both children and adults are interested in the same patches.
%Third, children may make rational, “educated” decisions for foragers of their smaller size and lesser strength. Children preferentially dig young ovy plants, whose tubers are small but shallow, while adults target deeper, larger tubers. Children exploit patches more thoroughly than adults.
%Fourth, Mikea children are neither pressured to bring home a full load of tubers, for they know adults will provision them, nor are they pressed for time when foraging, for they have little else to do. So while they probably do learn while foraging, they learn at their own leisurely pace. Their objectives when foraging may be primarily social and recreational.
%Fifth, despite the fact that Mikea children are probably not striving for efficiency, they approach the age of positive net production during adolescence, considerably earlier than Piro, Ache, Machiguenga, Ju/’hoansi San, and Hadza. Achievement of positive net production is more-or-less coincidental with the adoption of adult sexual division of labor and workload. Young people increase their foraging efficiency when opportunity costs increase.
%Ecological variation may be just as important as ethnographic generalization when constructing and testing theories about the evolution of childhood

\cite{gurven_how_2006} 
%Social explanations focus on intragroup competition, where extra time is necessary to develop social competency (Dunbar, 1998; Barton, 1999). The risk-aversion hypothesis, proposed by Janson and van Schaik (1993), argues that growth is slow among social primates in order to avoid resource competition and thereby serves to reduce the risk of dying due to fluctuations in food supply. The third hypothesis views optimal age at reproductive maturation as a trade-off between increased production from the benefits of growing longer (and hence larger) and the decreased probability of reaching reproductive maturity because with each additional unit of time invested in growth there is some risk of dying (Charnov, 1993). Finally, learning- and skills-based models focus on the difficult adult foraging niche of many primates, especially humans, where much time early in life is devoted to acquiring the critical coordination, skills, and knowledge necessary for proficient adult foraging (Bogin, 1997; Ross and Jones, 1999). The embodied capital (EC) approach extends this approach to explain delayed maturation, extended life span, and increased encephalization as a coevolutionary response to the demands of the difficult human foraging niche (Hill and Kaplan, 1999; Kaplan et al., 2000; Kaplan and Robson, 2002).
%This paper examines age trajectories of hunting ability and physical growth among a group of forager-agriculturalists, the Tsimane of Bolivia, in order to test several key differences between EC and GH with respect to limitations on adult productivity. The GH predicts that foraging success should be limited primarily by physical body size. Therefore, once mature adult size is reached, individuals are expected to quickly learn the necessary skills to become proficient adult foragers (Blurton Jones and Marlowe, 2002; Bird and Bliege Bird, 2005; Tucker and Young, 2005). Foraging success is thus limited by what Bock (2002) referred to as ‘‘growth-based’’ rather than ‘‘experience- based’’ capital. Alternatively, the EC predicts that because the human foraging niche is focused on nutrient-dense and difficult-to-acquire resources, long periods of learning and experience combined with physical growth are required to achieve foraging success (Kaplan et al., 2000; Walker et al., 2002; Bock, 2005; Gurven and Kaplan, 2006).
%More difficult tasks in terms of strength and skill should witness a later onset of peak productivity
%In this study, we disaggregate the components of hunting ability across the life span among the Tsimane of Bolivia in greater detail than in previous studies, using a combination of observation, interview, and experiment.
%There has never been much doubt that learning is an important component of childhood and teenage years. The controversy is about whether learning is a cause or consequence of delayed growth (Bock and Sellen, 2002).
%Cross-cultural variation in age-profiles of hunting also provides suggestive evidence that learning has played a major role in the determination of return rates. As modern hunting techniques are introduced, especially dogs and flashlights, the effects of skill seem to diminish.

\cite{crittenden_juvenile_2013}
%Here, we test the embodied capital model with naturalistic foraging and consumption data among juvenile Hadza hunter–gatherers of Tanzania to determine the extent to which children self-provision.
%When analyzing only food brought back to camp, age was not a significant predictor. When combining the amount of food back to camp and the amount consumed while out foraging, however, older children consistently collected more food. The data presented here suggest that although older children do appear to have greater overall foraging success, even very young children are capable of collecting a considerable amount of food. Our data, although lending support to the embodied capital model, suggest that although foraging efficiency increases with age, it remains difficult to determine if this efficiency is a byproduct of learning, strength, or perhaps individual motivation.
%One of the most striking characteristics of human life history is a protracted juvenile period of growth and development (Bogin, 1997; Leigh, 2001; Robson, van Schaik, & Hawkes, 2006). The embodied capital hypothesis argues that natural selection favors prolonged investment in growth and delayed reproduction because potential reduction in fertility (due to a late age at first birth) is superseded by the benefits of a long training period in which to learn difficult foraging tasks that may reduce adult mortality (Gurven & Kaplan, 2006; Kaplan et al., 2000). This model emphasizes the role of learning subsistence skills and argues that adult level foraging competence is limited by body size and accumulation of “brain-based capital”—skills and knowledge (Gurven & Kaplan, 2006).
%We also found striking sex differences in out-of-camp consumption. Male foragers, although bringing back fewer kilocalories to camp when compared to their female counterparts, consumed significantly more food while out foraging. Sex differences in consumption may be associated with sex differences in foraging strategies and/or household responsibilities
%The age of the child affected neither the distance traveled while out foraging nor the amount of food brought back to camp. When combining the amount consumed while foraging and the amount returning to camp, however, older children consistently collected greater amounts of food. These data, although lending support to the embodied capital hypothesis that predicts that an increase in age correlates with an increase in foraging proficiency, provide an interesting divergence from expected outcomes using the formal model. We suggest two important caveats to interpreting our data as direct support for the embodied capital model. First, a small number of foragers in the sample collected enough food to meet or exceed their own daily requirements on the days in which they foraged, which is counterintuitive to the embodied capital model that argues that children do not become net producers until late into their teen years (Gurven & Kaplan, 2006). Second, based on the type of data presented here, it remains difficult to determine if this efficiency is a byproduct of learning, strength, or perhaps individual motivation.
%An individual's daily returns can greatly fluctuate depending on food type and individual motivation to forage. The wide variation in overall return that characterizes our data suggests that in addition to differences in yield based on age or sex, the skill of the forager and the circumstances facing each individual are also mitigating factors influencing foraging success. Such wide variation in individual collection is consistent with published values on children's foraging on the Island of Mer (Bird & Bliege-Bird, 2005; Bliege-Bird & Bird, 2002), among the Martu of Australia (Bird & Bliege-Bird, 2005), and the Mikea of Madagascar (Tucker & Young, 2005).
