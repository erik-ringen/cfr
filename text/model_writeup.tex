% Options for packages loaded elsewhere
\PassOptionsToPackage{unicode}{hyperref}
\PassOptionsToPackage{hyphens}{url}
\PassOptionsToPackage{dvipsnames,svgnames*,x11names*}{xcolor}
%
\documentclass[
]{article}
\usepackage{lmodern}
\usepackage{amssymb,amsmath}
\usepackage{ifxetex,ifluatex}
\ifnum 0\ifxetex 1\fi\ifluatex 1\fi=0 % if pdftex
  \usepackage[T1]{fontenc}
  \usepackage[utf8]{inputenc}
  \usepackage{textcomp} % provide euro and other symbols
\else % if luatex or xetex
  \usepackage{unicode-math}
  \defaultfontfeatures{Scale=MatchLowercase}
  \defaultfontfeatures[\rmfamily]{Ligatures=TeX,Scale=1}
\fi
% Use upquote if available, for straight quotes in verbatim environments
\IfFileExists{upquote.sty}{\usepackage{upquote}}{}
\IfFileExists{microtype.sty}{% use microtype if available
  \usepackage[]{microtype}
  \UseMicrotypeSet[protrusion]{basicmath} % disable protrusion for tt fonts
}{}
\makeatletter
\@ifundefined{KOMAClassName}{% if non-KOMA class
  \IfFileExists{parskip.sty}{%
    \usepackage{parskip}
  }{% else
    \setlength{\parindent}{0pt}
    \setlength{\parskip}{6pt plus 2pt minus 1pt}}
}{% if KOMA class
  \KOMAoptions{parskip=half}}
\makeatother
\usepackage{xcolor}
\IfFileExists{xurl.sty}{\usepackage{xurl}}{} % add URL line breaks if available
\IfFileExists{bookmark.sty}{\usepackage{bookmark}}{\usepackage{hyperref}}
\hypersetup{
  pdftitle={Comparing child and adult foraging returns: a multilevel meta-analytic model},
  pdfauthor={Ilaria Pretelli, Erik Ringen, and Sheina Lew-Levy},
  colorlinks=true,
  linkcolor=black,
  filecolor=Maroon,
  citecolor=Blue,
  urlcolor=Blue,
  pdfcreator={LaTeX via pandoc}}
\urlstyle{same} % disable monospaced font for URLs
\usepackage[margin=1in]{geometry}
\usepackage{graphicx,grffile}
\makeatletter
\def\maxwidth{\ifdim\Gin@nat@width>\linewidth\linewidth\else\Gin@nat@width\fi}
\def\maxheight{\ifdim\Gin@nat@height>\textheight\textheight\else\Gin@nat@height\fi}
\makeatother
% Scale images if necessary, so that they will not overflow the page
% margins by default, and it is still possible to overwrite the defaults
% using explicit options in \includegraphics[width, height, ...]{}
\setkeys{Gin}{width=\maxwidth,height=\maxheight,keepaspectratio}
% Set default figure placement to htbp
\makeatletter
\def\fps@figure{htbp}
\makeatother
\setlength{\emergencystretch}{3em} % prevent overfull lines
\providecommand{\tightlist}{%
  \setlength{\itemsep}{0pt}\setlength{\parskip}{0pt}}
\setcounter{secnumdepth}{-\maxdimen} % remove section numbering
\usepackage[ruled,vlined]{algorithm2e}
\usepackage{setspace,amsmath,graphicx,pdflscape}
\doublespacing

\title{Comparing child and adult foraging returns: a multilevel meta-analytic
model}
\author{Ilaria Pretelli, Erik Ringen, and Sheina Lew-Levy}
\date{}

\begin{document}
\maketitle

\hypertarget{generative-model}{%
\subsection{Generative model}\label{generative-model}}

Assume that individuals go on foraging trips in which they successfully
acquire some return (\(y > 0\)) with probability \(1-p\) or come home
empty-handed (\(y = 0\)) with probability \(p\). Further assume that non
zero returns follow a log-normal distribution. Observed foraging returns
are thus mapped onto a ``hurdle'' model where:

\[ f(y) = \textrm{Bernoulli}(p)  ~~~~\textrm{if}~~y=0 \\ \]
\[ f(y) = (1-p) [\textrm{LogNormal}(\mu,\sigma)] ~~~~\textrm{if}~~y>0  \]

Previous studies of human foraging returns have found that both the
probability of a zero-return and the quantity of returns depends on
forager skill (\(S\)), which varies across the lifespan. As a directed
acyclic graph: \(\textrm{age} \rightarrow S \rightarrow p\) and
\(S \rightarrow \mu\). Koster et al.~(2019) modeled the relationship
between age and \(S\) as a concave downward function to account for
senescence among older adults. However, our focus was on the returns of
foragers below age 20---more than a decade earlier than the estimated
peak of foraging skill, so we did not model senescence. Otherwise, we
used the same functional form as Koster et al.~(2019) and Lew-Levy et
al.~(2021) to describe change in latent foraging skill with age:

\[S(\textrm{age}) = [1 - \textrm{exp}(-k\times \textrm{age})]^{b}\]

Where \(k\) is the constant rate of growth in foraging skill and \(b\)
is an elasticity parameter that determines the proportional change in
skill. \(b\) \textless{} 1 indicates diminishing returns (decreasing
differentials of skill with age), while \(b\) \textgreater{} 1 indicates
accelerating returns (increasing differentials of skill with age). Skill
itself may have nonlinear effects on foraging success, which we model as
an additional elasticity parameter \(\eta\). \(k\), \(b\), and \(\eta\)
were assumed positive, which means that skill is strictly increasing
with age and that higher skill always has a positive effect on foraging
returns. Finally, we add the log-linear \(\alpha\), which acts as an
intercept for foraging returns, independent of age.

\[ \mu = \textrm{log}(S^{\eta_{[\mu]}}\alpha_{\mu}) \]
\[ p = 2[\textrm{logit}^{-1}(S^{\eta_{[p]}}\alpha_{p}) - \frac{1}{2}] \]

Absolute values of foraging returns are not likely to be directly
comparable between or even within studies (if, for example the study
reported returns for two different resource types). However, the target
of our meta-analysis was not absolute production, but instead children's
productivity \emph{relative} to adults. Therefore we scaled returns
\(y\) by the mean adult return in a given study and resource type.

\hypertarget{measurement-error-in-adult-mean-return}{%
\subsection{Measurement error in adult mean
return}\label{measurement-error-in-adult-mean-return}}

We scaled child foraging returns by the mean return for foragers age
\(\geq\) 20. To propagate uncertainty in mean adult returns, for each
outcome in each study we treated it as a parameter
(\(y_{\mu[\textrm{Adult}]}\)) with known Gaussian error (the standard
error of the mean), such that:

\[ \mathbb{E}(y_{\textrm{[adult]}}) \sim \textrm{Normal}(\bar{y}_{[\textrm{Adult}]},\sigma_{\bar{y}_{[\textrm{Adult}]}}) \]

One complication is that the sample mean and standard error of the mean
will be deflated by the inclusion of zero return values. Note that the
expectation of \(y\) under our generative model is:

\[ \mathbb{E}(y) = (1 - p)[\textrm{exp}(\mu~+~\frac{\sigma^2}{2})] \]
\[ \textrm{Var(y)} = [\textrm{exp}(\sigma^2) - 1] \textrm{exp}(2\mu + \sigma^2) \]

Because we want to partition zero and non-zero returns among children,
we should only scale by the non-zero adult returns. The sample mean
\(\bar{y}\) will always be less than the quantity we actually want
\(\bar{y}|y>0\). But the average of non-zero returns can be recovered:

\[\bar{y}|y>0 = \frac{\bar{y}}{(1 - p)}\] If an average \(\bar{p}\) is
not available for adult data, we use model predicted
\(p|\textrm{age} = 20\) as a rough estimate.

Similarly, to adjust the standard error of the mean so as to remove the
effect of zero-returns, we estimate:

\[ \sigma_{\bar{y}|y \geq 0} = \sqrt{\frac{\textrm{var}(y|y >0)}{n}}\]

\[ \textrm{var}(y|y>0) = \frac{\textrm{var}(y)}{(1 - p)} - p[\bar{y}|y>0]^2 \]
Where \(n\) is the sample size among adults, \(\bar{y}|y>0\) is the same
as given above and \(\textrm{var}(y)\) is \((\sigma_{\bar{y}}^2)n\).
Therefore we can adjust biased adult estimates of foraging returns in
light of zero-return rates derived from the data.

\hypertarget{integrating-individual-level-data-with-study-level-summary-statistics}{%
\subsection{Integrating individual-level data with study-level summary
statistics}\label{integrating-individual-level-data-with-study-level-summary-statistics}}

Our data included a mix of individual-level returns (e.g., a forager
brought back \(y\) kilograms of fish) and summary statistics, such as
the mean and standard deviation of returns for children/adults (e.g.,
over the study period, children averaged \(\bar{y}\) kilograms of fish
per day). The challenge was to synthesize two distinct types of data:
observations drawn from the random variable \(f(y|\mu,\sigma)\) and
observations drawn from the random variables \(f(\mu)\) and
\(f(\sigma)\).

When returns were given as summaries, we first adjusted for deflation
due to zero-returns as described above. Then, to map these adjusted
sample estimates onto the log-normal parameters \(\mu\) and \(\sigma\)
we used the method of moments:

\[ \mu_{[\textrm{obs}]} = \textrm{log}[\frac{(\bar{y}|y>0)^2}{\sqrt{\textrm{var}(y|y>0) + (\bar{y}|y>0)^2}}]\]
\[ \sigma_{\textrm{[obs]}} = \sqrt{\textrm{log}[\frac{\textrm{var}(y|y>0)}{(\bar{y}|y>0)^2} + 1]}\]

Then we can update the likelihood:

\[ \mu_{[\textrm{obs}]} \sim \textrm{Normal}(\mu,\sigma_{\mu[\textrm{obs}]}) \]
Where \(\sigma_{\mu[\textrm{obs}]}\) is the standard error of
\(\bar{y}|y>0\) after the method of moments transformation.

(NOTE: not sure how we would meta-analyze the variance terms given
summary statistics\ldots)

\hypertarget{what-if-no-adult-returns-are-available}{%
\subsection{What if no adult returns are
available?}\label{what-if-no-adult-returns-are-available}}

Not sure! Need feedback.

\hypertarget{adding-predictors}{%
\subsection{Adding predictors}\label{adding-predictors}}

The meta-goal of this meta-analysis was to understand why some authors
report that children are relatively bad at foraging, while others argue
that children are proficient foragers from young ages.

We allowed the parameters \(k, b, \eta\), and \(\alpha\) to vary among
and between studies (i.e., if a single study had multiple outcomes) and
resource type using random effects. We also assessed sex differences
with an overall average sex difference and outcome-specific deviations
as random effects.

\[ \textrm{log}(\alpha) = \alpha_0 + v[\alpha_0,\textrm{outcome}] + v[\alpha_0,\textrm{resource}] + (\alpha_{[\textrm{sex}]} + \alpha_{[\textrm{sex,outcome}]})*\textrm{sex} \]
\[ \textrm{log}(k) = k_0 + v[k_0,\textrm{outcome}] + v[k_0,\textrm{resource}] + (k_{[\textrm{sex}]} + k_{[\textrm{sex,outcome}]})*\textrm{sex} \]
\[ \textrm{log}(b) = b_0 + v[b_0,\textrm{outcome}] + v[b_0,\textrm{resource}] + (b_{[\textrm{sex}]} + b_{[\textrm{sex,outcome}]})*\textrm{sex} \]
\[ \textrm{log}(\eta) = \eta_0 + v[\eta_0,\textrm{outcome}] + v[\eta_0,\textrm{resource}] + (\eta_{[\textrm{sex}]} + \eta_{[\textrm{sex,outcome}]})*\textrm{sex} \]

We also model correlations between the random (varying) effects \(v\) to
account for, for example, the possibility that studies where the base
rate of skill acquisition is higher may have lower age-independent
returns.

\hypertarget{uncertainty-in-age}{%
\subsection{Uncertainty in age}\label{uncertainty-in-age}}

Forager age was not reported exactly in any study. Most frequently,
authors reported an integer age for each child. In other cases an
interval of possible ages was given (e.g., 4-7). Depending on the
reported uncertainty, we modeled age with either Gaussian or uniform
interval measurement error.

\hypertarget{uncertainty-in-sex}{%
\subsection{Uncertainty in sex}\label{uncertainty-in-sex}}

In cases where sex of the forager was not reported, we average over sex
differences using log\_mix() in proportion to how often males and
females appeared in a given study.

\hypertarget{individual-differences}{%
\subsection{Individual differences}\label{individual-differences}}

Some studies have multiple measurements from the same forager, in which
case we include random effects to account for non-independence of the
data and estimate between-individual variance in returns.

\end{document}
