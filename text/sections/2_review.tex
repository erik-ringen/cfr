\section{Foraging for children research: a review}

We review here the existing literature concerning juvenile foraging activities, adopting a combination of historical and thematic approach. We start by describing how children foraging has come to the attention of researchers during the second half of the 20th century. We then follow by discussing some of the main subjects interested by data on children foraging, highlighting the areas in which more quantitative information would be important, for example to answer question about human evolution or to design policies in changing environments.

\subsection{Becoming adult}

Despite ethnographic or descriptive account of children foraging, quantitative research on these activities had a very slow development. %INPUT cite Mead and others?

Among the first works focusing on children in small scale societies, \cite{lee_social_1976} strikes for how it depicts children as economically dependent from their families. Collecting time allocation data among the !Kung hunter gatherers of the Kalahari desert, she deduces that young individuals barely engage in productive activities independently from their parents. This is explained in part with the technologies available, and in part with the distribution of resources in the environment. The technologies !Kung employ to forage, according to Draper, are too simple and do not facilitate difficult tasks enough for children to engage with them. Also, low predictability in time and space of food resources contribute to excluding children from productive activities out of camp. 

In the following decade, \cite{nag_anthropological_1978} and \cite{munroe_childrens_1984} (among others?) also measure time allocation of children. Focusing on village and domestic settings, these papers offer a slightly different account of children work. In the societies they observe, children as young as three years old are reported spending at least a part of their time in productive/household activities. These include childcare and garden cultivation, but remarkably not foraging. Such an absence might be due to the fact that the studies are conducted in agricultural societies ('paesant societies' according to Nag et al.), but also to the method of data collection, that appear not to allow for observation in forests or other places where children would forage.

Up to the late '80s, the only paper reporting reporting quantitative accounts of children foraging is \cite{kawabe_development_1983}. This paper, though, which describes Gidra boys hunting and fishing in the lowlands of Papua New Guinea, goes almost unnoticed in the contemporary literature. 

\subsection{Hadza vs !Kung: renewing the approach on children foraging returns}

It is understandable, then, that researchers approach children foraging activities expecting low engagement in this activity. For example, \cite{blurton_jones_why_1997}, titled "Why Do Hadza Children Forage?", list different hypothesis of why indeed children would forage in a hunter gatherer population. 

In particular, in the literature of the early '90s, the question of why would children take up foraging is framed as a comparison between the !Kung and another African hunter gatherer population, the Hadza. Contrary to !Kung, Hadza children venture in the scrubland surrounding their villages in small mixed age groups to dig tubers, hunt small game or pick fruits.  \cite{hawkes_hadza_1995}, p. 694, for instance, states: "The Hadza are an exception to the conventional anthropological wisdom that the children of mobile hunter-gatherers do little to support themselves but instead rely completely on their parents for subsistence". 

The contrast is sharpened by the fact that !Kung and Hadza are similar under various aspects. Considered remaining examples of the hunter gatherers inhabiting Eastern and Southern Africa before the Bantu expansions, both populations largely rely -or relied until recently- on wild products for food and shelter. They live in mobile camps following the availability of water and other resources, and have traditional diets comprising mainly tubers, nuts and fruits and wild meat (citation needed).

But the differences, in this case, count more than the similarities.
\cite{blurton_jones_foraging_1994, blurton_jones_differences_1994} describe the different hardships offered to Hadza and !Kung by their environment. Despite both populations live in arid, Savannah-like climate, Northern Tanzania, where the Hadza live, is a more favorable habitat. The landscape is more varied, reducing the chances for inexperienced children to get lost, and the risk offered by predators or other adversities is considered lower by Hadza parents. But the main factor seems to be the distance of resource patches from camp. Hadza children have usually access to baobab trees within few hundred meters from camp. When the trees produce fruits, the children can collect up to half of their daily caloric needs by picking them. In order to collect a comparable amount of calories, !Kung children should cover more than five kilometers to the closest Dobe nut grove. For the sake of comparison, no Hadza child goes that far from camp without adult supervision. Moreover, children do not follow their parent in the excursions to these groves, as adults report that the lack of water and shade along the way makes children more of a burden than help. 

In the same paper, Blurton Jones and colleagues suggest that for children to accompany the adults on these long trips is actually sub-optimal at the household level. Children appear to spend their time more productively by processing the food collected by their mothers, rather than collecting it themselves. More in detail, \cite{hawkes_hadza_1995} calculate the costs and benefits of different tasks children could accomplish, comparing data on various activities payoffs for the mother-offspring team. Beginning "with the simple hypothesis that children seek to maximize their mean rate of nutrient acquisition while foraging" (p. 694), they find that "Just as Hadza mothers earned higher team rates by taking their children to the distant berry patches than by foraging closer to camp, so !Kung mothers earn higher team rates by not taking their children to the mongongo groves, leaving them at home to crack nuts instead" (p. 697).

From this in comparative analysis, emerges as the main message that children can behave optimally and that what is optimal depends on the environment and resources available. The work on juvenile foraging that follows tries to increase the amount and variability of data available, to understand how children and teenagers face cost-benefit trade offs in different conditions, as well as the evolutionary implications to foraging behaviors.

\subsection{A children's world}

In particular, in the early 2000s, Rebecca Bliege Bird and her husband Douglas Bird published a series of papers on children foraging activities. For the first time, what children do was observed from their own point of view: "Paying attention to the differences from a child's perspective may help adults remember what being a child was all about." (\cite{bird_children_2002}, p292).

Working first on the island of Mer, in the Torres Strait Archipelago, and later on among the Mardu of Western Australia, the couple tried to understand the factors defining the trade offs children face when foraging. In both societies, children leave often the camp to hunt or fish, and are often collecting a large proportion of their daily caloric need ("Meriam children under age 15 each provided an average of 750 kcal/day", \cite{bliege_bird_children_1995}, p 16).

In \cite{bliege_bird_children_1995} and later more in detail in \cite{bird_ethnoarchaeology_2000} and \cite{bird_children_2002}, the authors observe how children and adult foraging behaviors differ on the intertidal reefs of Mer Island. They argue that the difference in the proportion of shell species collected is a function of adult vs. child body size, and that this difference is consistent with predictions from prey-choice models. 
%These models predict whether or not a prey should be pursued on encounter. They are used to understand decision making during foraging, because some preys carry a premium lower than the cost paid to capture/process/transport that prey, and thus are not worth collecting. Changes in diet composition would imply, according to this model, a change in the cost benefit ratio of certain species, or reduction in the abundance of others. In this case,
Specifically, children are smaller and walk at lower speed along the reef, which reduces their chance of encountering high value prey. It becomes optimal, then, to include lower value preys in the diet, and hence we observe the appearance of the small \textit{Trochus} and \textit{Strombus} in the children bounty, which are not collected by the adults.
%The \cite{bird_ethnoarchaeology_2000} paper is especially interesting, as it explicitly tests predictions from the prey choice model, predictions which are largely confirmed by real data. This has implications for the interpretation of archaeological data, as similar patterns could be originated as an effect of children foraging, as an alternative to other mechanisms such as a reduction in the abundance of certain species.
Similarly, in \cite{bird_mardu_2005}, Mardu children chose how to spend their foraging time in accordance with their size. Being slower than their parents, it is advantageous for them to hunt the smaller but more common lizards which live in rocky patches close to their village, instead of traveling to further patches where adults hunt. This way, they increase their total returns, by both reducing energy and time spent in traveling, and encountering preys at a higher rate 
%(as well as reducing the risk of dehydration or heat strokes and making time to care for their younger siblings). 
Other factors defining what children forage might have nothing to do with their strength or size, but instead with what adults choose to prioritize. Social status or apparent reproductive value have different fitness consequences for parents and offspring, translating ultimately in different decisions in front of trade offs (\cite{bird_constraints_2002}). This has been used to explain big game hunting, which in some hunter gatherer societies has lower return rates than other less prestigious activities, but is still performed by individuals trying to increase their social capital or to impress a potential mate (citation needed \cite{}).   

By paying attention to children foraging, then, we can not only better understand the factors directing choices between trading off options, but also test models regarding foraging behaviors under a variety of assumptions.

\subsection{Evolution of childhood}
During the '90s and early 2000, new attention to children foraging returns has been brought by a growing literature discussing the evolution of childhood (\cite{bogin_evolutionary_1997}).%more citations.

In particular, Kaplan and colleagues developed a theoretical framework which would explain the emergence of several aspects characterizing human life history as a coevolution with 'dietary shift toward high-quality, nutrient-dense, and difficult-to-acquire food resources.' (\cite{kaplan_theory_2000} p.156, and see \cite{kaplan_theory_1996, kaplan_evolution_1997, kaplan_embodied_2001, kaplan_emergence_2002, kaplan_embodied_2003, kaplan_neural_2003, kaplan_life_2006, kaplan_evolution_2007}). %Namely, 'an exceptionally long lifespan, an extended period of juvenile dependence, support of reproduction by older postreproductive individuals, and male support of reproduction through the provisioning of females and their offspring' would have been selected to allow the exploitation of a highly complex foraging niche. High return food packages, such as big game, but also tubers and hard-to-get resources, would allow to support dependent offspring, but also require the development of foraging abilities over a prolonged period of time. 
This framework, often referred to as the Embodied Capital Hypothesis, implies that individuals increase their capacity to extract resources from the environment as they build up the set of traits that allows them to do so (skills, strength, knowledge etc, in general referred to as embodied capital, from human capital theory in economics), until when they are able not only to collect enough food for their own energetic requirements, but also to produce a net energetic surplus to feed their dependent children or even grandchildren. To test this hypothesis, then, data on how food production changes along the lifespan must be collected, with a focus on the pre-reproductive period.

%In \cite{kaplan_evolution_1997}, to exemplify the hypothesis, the average productivity per age is compared with the consumption for three different societies. But with more attention to individual level returns,
In the early 2000s, several authors working on children behavior finally set up to doing so. 

A special issue of Human Nature, published in June 2002, reports discussions on the evolution of childhood with data from four different populations. \cite{bird_children_2002} collected naturalistic return data from children collecting seashells on the island of Mer, and also, in \cite{bird_constraints_2002}, from young Mardu hunging lizards in the Western Australian desert. \cite{bock_learning_2002} observed and measured both food collection and processing in a multi-ethnic community in the Okavango Delta of northwestern Botswana. And, finally, \cite{blurton_jones_selection_2002} organized competitions to measure the development of a combination of foraging-relevant techniques among the Hadza in Tanzania. 
Also in 2002, \cite{walker_age-dependency_2002} focused on hunting returns among the Ache of Paraguay with a combination of naturalistic observations, contests and recording of activities on a diary by informants. 
In 2005, \cite{bock_what_2005} published again on the same subject, and \cite{tucker_growing_2005} contributed with data on tuber digging among the Mikea hunter gatherers of Madagascar. 
\cite{gurven_how_2006}, combining interviews and naturalistic observations, reported hunting returns among the young and adult Tsimane horticulturalists of Bolivia and, most recently, \cite{crittenden_juvenile_2013} described children foraging for various products among the Hadza with naturalistic observations of returns.

The majority of these papers shares some similar features. 
First, they set up to discuss different hypotheses that can explain the evolution of childhood, often contrasting the Embodied Capital Hypothesis with some other possible mechanism underlying the elongation of pre-reproductive period in humans. 

Second, they often remark on the complementary contributions to foraging success of two types of embodied capital. Physical abilities, such as strength, coordination or speed, depend on the development of the body and constitute a growth-based capital, while knowledge, skill and intuition are an experience-based capital, which requires time and implementation to be acquired. 
The data are deemed to hold different patterns according to the alternative hypotheses on how they are generated and to the interpretation the authors make of these hypotheses. For example, time spent in school, depriving children of the opportunity to practice, should have the effect of reducing returns depending on experience-based capital (\cite{blurton_jones_selection_2002}).
The conclusions these papers reach are not consistent, but some general messages can be extracted. 
%AND WE WILL TREAT THE IMPLICATIONS IN GREATER DETAIL IN THE THIRD SECTION OF THIS PAPER.

Mainly, the way returns change with age depends on the specific constraints of the activity taken into consideration. For example, hunting appears to be a particularly skill intensive activity, with multiple combining elements each depending on different experience-based forms of capital (finding prey, tracking, archery skills), but also requires some strength and endurance. The data suggest that hunting might be one of the foraging activities for which returns peak later in life. 

Hunting is treated remarkably clearly by \cite{koster_life_2020}, with a large cross cultural database of juvenile and adult returns for more than 1,800 individuals from 40 societies, spanning several years. Age at peak hunting success is in general quite high, with an average between 30 and 35 years of age, but it also varies a lot across societies. Environmental variability in the nature as well as the frequency of encounter with prey might be at the root of the differences, further reinforcing the idea that foraging success depends on the specific condition of the task.
%INPUT maybe we should expand on Jeremy and Richard's paper, or we do more when describing the model?

\subsection{Cooperative breeding, pacing of reproduction and self sufficiency}

Children productive activities have repercussions on other aspects of human life history. \cite{hawkes_hadza_1995} demonstrated that the dyad mother-offspring maximizes joint returns when foraging together. On a higher level, children production has consequences for the whole household, reducing the cost of rearing each child through self-support and often also contributing to feeding younger siblings when adults are absent. Although in most societies individuals don't reach self sufficiency until when they are in their late teens or early twenties (\cite{kaplan_evolution_1997}), and only later on produce enough surplus to support a family, the fact that children forage might allow the peculiarly human pattern of reproduction with multiple dependent offspring. This subject has been widely explored with time allocation data in \cite{kramer_variation_2002, kramer_maya_2005, kramer_childrens_2005, kramer_does_2009} and elsewhere, and also partially discussed in \cite{crittenden_juvenile_2013} and \cite{bird_constraints_2002}, but would deserve more attention with the collection of individual and family level data on production and consumption. 

\subsection{Learning and knowledge}
A different area of research that has demonstrated interest for children foraging activities is the one studying learning and development of knowledge in small scale societies (\cite{gallois_local_2017, koster_wisdom_2016, lew-levy_how_2017, lew-levy_who_2019, reyes-garcia_adaptive_2016, setalaphruk_childrens_2007}). Adult individuals need a vast array of information to navigate their social and ecological environment. A good part of this knowledge is relevant for foraging, so that a big part of the field focuses on understanding the transmission and measuring ecological knowledge at different ages. Investigating the role of environmental information for practical purposes, \cite{koster_wisdom_2016} measures knowledge relevant for fishing activities in individuals of at least 10 years of age in a Mayanga village in Nicaragua. He also looks for, and fails to find, correlation between these measures and estimated fishing abilities. \cite{reyes-garcia_adaptive_2016} on the contrary finds out that local ecological knowledge provides individual returns in terms of hunting yields. In light of the increasing interest for local knowledge alongside with learning strategies in small scale societies, more work on how children employ the acquired knowledge when foraging would be very useful. 

\subsection{Diet}
By foraging autonomously, children have the potential to integrate the diet they receive at home. In poorer and marginalized settings, especially, the contribution of foraged wild foods can be important (add citations) and children can procure these foods by themselves, even as side consequence of playing activities. Moreover, individuals, as they grow, often have dietary needs that differ from those of the adults and might include foods that are not appreciated or provided by caregivers (citation search things on unripe mangoes for example) which they could have access by searching individually.  
Despite the potential relevance of wild foods in nutritional status, the data on children involvement in foraging activities are scant in non hunter-gatherer settings. 
\cite{mcgarry_children_2009}, as an example, observe an increase in children's diet quality and variability thanks to foraged food in rural South Africa, especially in poorer households. On the contrary, \cite{lee_childrens_2009} does not find associations between child foraging and nutritional benefits in a Mexican shanty town. 
Studying children contribution to diet, is particularly important in societies undergoing dietary transitions. A number of health concerns is associated with changes in availability of agricultural and then processed foods (\cite{satia_dietary_2010}). Focusing on how transition to domesticate foods affect foraging patterns of Hadza children, \cite{pollom_changes_2020} report a reduction in the variety of wild foods collected between 2005 and 2017. 

The implications of children and teenagers foraging activities for the development of policies to improve health and education in marginalized areas would call for more effort in these kind of studies.

\subsection{Emergence of sexual division of labor}
Finally, observing children foraging activities is necessary to understand the emergence of sexual division of labor. In many hunter gatherer populations, the two sexes target different foods when foraging as adults, but have similar foraging patterns as children. We can get some insight from time allocation data, which show females devolving more time to household and childcare tasks as they age, whereas males spend more time further from home (\cite{froehle_physical_2019}, and Lew-Levy (hopefully in press)). But specific analyses trying to understand when the gender specialization in foraging emerges are missing.

\subsection{Conclusions}
The widespread prevalence and extensive implications of children foraging behaviors would call for more attention to this phenomenon. Children based research raises multiple specific concerns, not least the ethical implications of working with minors, but the possible benefits, especially because concerning marginalized societies, can outnumber the difficulties, as long as research is appropriately conducted.

In the following section, we use data coming from many of the studies cited above to delineate a general profile of children foraging behavior. We hope this can be used as a background for future research expanding on the themes presented here. We also provide some indications to make children foraging research more comparable and better at contributing to variety of scientific questions.


%INPUT please, check these sections out, if you think an extension of any would be advisable, or sections are missing, you can proceed or let me know and I'll do my best. Comment out passages if you think it would be advisable to shorten the text.