\section{Introduction}
All around the globe, children search their environment looking for and collecting food. 
They can be berry picking on summers, or consistently collect half their caloric needs: no matter how much their diet is composed of wild food, children forage. 
Information on how, when and how much young individuals forage is a contribution to different disciplines studying human behavior.

%ADD CONNECTIVES AND CITATIONS
%insight in trade off related to foraging given that children have different constraints
For instance, the specific constraints and needs of children and teenagers can help understand how humans make foraging decisions over cost-benefit trade offs in terms of energy, time, social status, etc. (\cite{bliege_bird_children_1995, bird_ethnoarchaeology_2000, bird_children_2002}).
Or, also, the study of nutritional status and requirements in various ecological settings needs to understand what food children and adolescents produce, as this is an important component not only of their diet (\cite{pollom_changes_2020, pollom_effects_2020, mcgarry_children_2009}),
but also of that of their families (\cite{fouts_who_2009}). 
%\cite{mcgarry_children_2009}However, 62% of all the children interviewed (both non-school and school-going) supplemented their diets with wild foods. Dietary diversity showed a 13% increase when wild-food supplementation occurred. 
%\cite{fouts_who_2009} "Child feeding involvement by juvenile relatives (e.g., siblings and cousins) was not predicted by family transition, birth order, or physical ecology. However, it is striking to note that juvenile relatives provided food as much as adult female relatives, and nearly as much as elderly female relatives. Since birth order did not predict levels of child feeding by juvenile relatives, one could assume that even if a child is first born, with no older siblings, they are still receiving care from cousins. This is not surprising since the Aka live in large extended family groups and juveniles typically make up at least half of the population. However, the lack of birth order effect is somewhat contradictory to evolutionary hypotheses (see Turke, 1988) that suggest siblings in particular will provide care due to their close genetic relatedness compared to other alloparents with more distant genetic relatedness (e.g., cousins)." 
%Borgerson, Cortni, et al. "An evaluation of the interactions among household economies, human health, and wildlife hunting in the Lac Alaotra wetland complex of Madagascar." Madagascar Conservation & Development 13.1 (2018): 25-33. for indication that supplementing diet could reduce usage of wild foods
The fact that young individuals have a double role as producers, as well as consumers, has obvious implications for household economies, which can count on incoming food from each child, as well as for reproductive decisions of their parents, which can rely on their offspring feeding their siblings when performing childcare duties (\cite{kramer_variation_2002, kramer_maya_2005, kramer_childrens_2005, kramer_does_2009, crittenden_juvenile_2013}). 

As a consequence of being interrelated at these multiple levels with human behavior, children foraging activities have probably contributed to the evolution of several modern humans life history traits. In order to learn the complex set of skills required to reach full foraging potential, natural selection might have promoted an extension of the pre-reproductive period in a prolonged condition of at least partial dependency (\cite{kaplan_theory_2000}).
At the same time, the fact that this dependency is not complete, as children procure part of their own food and contribute to household workload, probably allowed the characteristically human trait of having overlapping dependent offspring (\cite{kramer_childrens_2005}).
%cite

Moreover, as foraging children carry on traditional subsistence practices while their parents engage in market labor elsewhere ,
they can engage in the acquisition, and thus preservation, of traditional ecological knowledge (\cite{setalaphruk_childrens_2007}).
%cite
Given the applied nature of these information, tracking how knowledge is acquired and used can give important insight on learning and teaching strategies in traditional societies, as well as on psychological mechanisms underlying these processes (\cite{lew-levy_who_2019}).
%cite
%For example, \cite{chpeniuk} observed that individuals who foraged a higher breath of goods have a better sense of biodiversity as adults.
%studies on how children interpret and conceptualize the natural environment often contrast data from groups where children have different levels of reliance on wild foods.
%cite paper on hunting in the us, and have a look at katja leibal and annie wertz work

%more on psychology?

%effect of changing environments: if children foraging is important, and an environmental change makes it impossible, is there an effect on health? 
%Also, schooling: if children go to school and do not hunt, for example, how is this supplemented in their diet? Schools provide meals, but sometimes cheap carbs (Tanzania liquid porridge)

%info for archaeology and other fields




% Lack of review/comparative analyses
Despite the relevance of this behavior for different fields, relatively few researchers have collected return data from children foraging. Comparative works are especially lacking and, as a result, little can be said on general trends. 

% lack of focus on children, necessary to engage with them specifically, 


\vspace{0.5cm}
% we're addressing this void by reviewing the literature and doing a meta analysis of available data (FOCUS on children data, various types of data) thanks to bayes analysis
With this paper, we aim, first, at providing an historical overview of researches describing children foraging activities. 
Specifically, we review how the perception of children foraging changed since the phenomenon has been first described quantitatively; we summarize the main contributions of research on children foraging results and describe some of the main papers that report this kind of data.

Second, we provide a comparative analysis of foraging returns across XX different societies, tapping on previously published data available in the literature. 
We analyse how trajectories of foraging returns change with age, across different resource types. To do so, we developed a Bayesian model inspired by \cite{koster_life_2020} which allows to understand how the factors determining increasing returns with age contribute to the total variation. 
To address the difficulties of comparing different data sets, within we focus on the change of foraging returns of children and adolescents (with an arbitrary cut off at 20 years of age), relative to adult values.
%difficulties: compare data on different resources,

Finally, we focus on one hypothesis tested in several of the analyzed papers, that human early life history has been modeled to allow for learning of complex skills \cite{kaplan_theory_2000}, and look for correspondences on both the data and the conclusions drawn from them across papers.  


% paper structure
%   -review
%   -present the model
%   -metaanalysis 
%   -comparison of papers addressing the same hypothesis
%   -conclusions




%BIN

%Trends of  foraging return increases with age relative to average adult value have been modeled taking into account variation in resource collected, society of provenance and method of data collection. 
%Foraging requires several skills and imposes complex trade offs, that inevitably change as children grow and their bodies change. 

%foraging important for HBE and developmental trajectories interesting
%Foraging activities have been a central theme in Human Behavioral Ecology since its conception (\cite{laland_sense_2011}, chapter 4). This is due, in one hand, to the assumption that foraging returns are associated with reproductive fitness, which makes them relevant for evolutionary studies ("Consider a situation in which there is a direct relationship between time spent hunting, the amount of game acquired (measured in total weight or its equivalent in calories or nutrients), and the relative reproductive success of the hunter." in \cite{hawkes_how_1985}). And, in the other hand, the fact that returns can be clearly quantified put them at the center of several models producing testable predictions, as in optimal foraging theory (for a review, see \cite{smith_anthropological_1983}). 

%Later, researchers started focusing on the development of foraging skills, bringing attention to age specific returns. For example, \cite{koster_life_2019} developed a model that traces age trajectories of individual hunting returns in 40 study sites. 

%At the same time, the work of young individuals in subsistence societies became apparent through quantitative and qualitative analyses.
 
 
%why is the analysis of children behavioral returns important (evolution of life history, learning and mechanisms limiting returns, children welfare -e.g. nutritional status, social role and schooling-, impact of ecological and social change, all need info on children work, role in the family and also what and how much they forage.) 
%Understanding how children and adolescents interact with their environment to extract food resources has implications in various areas of the study of human behavior. 
%For example, nutritional status and children welfare are directly affected by their foraging activities: is the food they procure an important contribution to their caloric intake? Are the socializing opportunities that happen during this activity important for the development of individuals? The effects of social innovations, such as the increase in schooling time, or the ecological changes can have important impacts on children nutritional status. 
%Also, Being a proficient adult in any human society requires a vast array of skills, that are necessarily learnt over time. How returns change along the life of individuals can help clarify how this process of skill acquisition works.

%in order to do this, we need to better understand how children improve their foraging returns, and what contributes to this change, and also understand if there are general trends 
