\section{Introduction}
Children, all around the globe, search their environment looking for and collecting food. 
They can be berry picking on summers, or consistently collect up to half their caloric needs: no matter how much their diet is composed of wild food, children forage. 

Information on how, when and how much young individuals forage is a contribution to  different disciplines studying human behavior.

The food that children and adolescents produce, for example, is sometimes an important component not only of their nutrition, but also of that of their families. This has obvious implications for household economies, and should be taken into consideration 





%foraging important for HBE and developmental trajectories interesting
Foraging activities have been a central theme in Human Behavioral Ecology since its conception (\cite{laland_sense_2011}, chapter 4). This is due, in one hand, to the assumption that foraging returns are associated with reproductive fitness, which makes them relevant for evolutionary studies ("Consider a situation in which there is a direct relationship between time spent hunting, the amount of game acquired (measured in total weight or its equivalent in calories or nutrients), and the relative reproductive success of the hunter." in \cite{hawkes_how_1985}). And, in the other hand, the fact that returns can be clearly quantified put them at the center of several models producing testable predictions, as in optimal foraging theory (for a review, see \cite{smith_anthropological_1983}). 

Later, researchers started focusing on the development of foraging skills, bringing attention to age specific returns. For example, \cite{koster_life_2019} developed a model that traces age trajectories of individual hunting returns in 40 study sites. 

At the same time, the work of young individuals in subsistence societies became apparent through quantitative and qualitative analyses.
 

% Lack of review/comparative analyses
Despite this diffused attention, though, few researchers have collected data on children foraging behavior. Comparative works are especially lacking and as a result, little can be said on general trends. 

%why is the analysis of children behavioral returns important (evolution of life history, learning and mechanisms limiting returns, children welfare -e.g. nutritional status, social role and schooling-, impact of ecological and social change, all need info on children work, role in the family and also what and how much they forage.) 
Understanding how children and adolescents interact with their environment to extract food resources has implications in various areas of the study of human behavior. 
For example, nutritional status and children welfare are directly affected by their foraging activities: is the food they procure an important contribution to their caloric intake? Are the socializing opportunities that happen during this activity important for the development of individuals? The effects of social innovations, such as the increase in schooling time, or the ecological changes can have important impacts on children nutritional status. 
Also, Being a proficient adult in any human society requires a vast array of skills, that are necessarily learnt over time. How returns change along the life of individuals can help clarify how this process of skill acquisition works.

%in order to do this, we need to better understand how children improve their foraging returns, and what contributes to this change, and also understand if there are general trends 

% we're addressing this void by reviewing the literature and doing a meta analysis of available data (FOCUS on children data, various types of data) thanks to bayes analysis

With this paper, we aim, first, at providing an historical overview of researches describing children foraging activities. 

Second, we provide a comparative analysis of foraging returns across XX different societies, tapping on previously published data available in the literature. 
We analyse how trajectories of foraging returns change with age, across different resource types. To address the difficulties of comparing different data sets, within we focus on the change of foraging returns of children and adolescents, (with an arbitrary cut off at 20 years of age, relative to adult values.

Finally, we focus on one hypothesis tested in several of the analyzed papers, that human early life history has been modeled to allow for learning of complex skills \cite{kaplan_theory_2000}, and look for correspondences on both the data and the conclusions drawn from them across papers.  
% paper structure
%   -review
%   -present the model
%   -metaanalysis 
%   -comparison of papers addressing the same hypothesis
%   -conclusions




%BIN

%Trends of  foraging return increases with age relative to average adult value have been modeled taking into account variation in resource collected, society of provenance and method of data collection. 
%Foraging requires several skills and imposes complex trade offs, that inevitably change as children grow and their bodies change. 
