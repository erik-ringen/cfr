\section{Data selection and extraction}
\subsection{Paper selection}
% Searched the scopus, wiley, science direct, google scholar and some other with keywords "children"&"foraging"&"returns“
% Read all abstracts and selected a subset that had info on children or foraging
% These read more in detail, by two different people, to see if there are data
% Of those with data, scanned bibliographies and CVs of first authors to see if there are relevant papers
% A selection of papers containing some sort of data on foraging returns was made (To our knowledge, an exhaustive list )

\subsection{data selection}
% Of the papers containing data, we extracted the metadata from the text
% This allowed to check for repeated data sets and for quality of data, for example the Kawabe paper was not adequate.
% At the moment we have 20 papers, with data from populations (size of the dot is number of studies from it, with Hadza data present in 7 studies from 1988 to this year)

\subsection{data extraction}
%copied dataframes, used package for images, obtained some original data

\subsection{problematics}
% Data are different
% Different units of measure: some are kcal per hour, some grams per hour, some are per foraging trip and some even different. 
% Also protocols are different, one study is experimental, most are observations but with different levels of details (focal follows, or weighing when returning to camp). 
% To make them comparable, we are analyzing them as proportion of adult foraging returns. 
% Many of the papers have data across the life span, so we are averaging above 20 and comparing to that. 
% If the paper has no adult returns, we are looking for adult returns from the same population and resource, shortly, comparable adult data.
% Sex is not always reported, in the model we input proportion of males (usually present) and estimate returns based on that
% Each paper has different problems that we try to take care in the script to extract data
