\section*{Abstract}
Children forage in a vast array of societies, spanning from traditional hunter gatherer groups, to Mexican shanty towns. Researchers have repeatedly measured this behavior, mainly trying to explain the delayed age at first reproduction in our species with a need-to-learn hypothesis. But data on children foraging success is sparse and difficult to compare, and inferences on the evolution of early life history traits hard to draw. With this paper we aim at summarizing available return data in a single coherent meta-analytical framework and describe the curve of increases in foraging return with age. Moreover, the multilevel Bayesian model we employ allows to estimate the latent skill underlying foraging returns. We also consider different resources, starting from the assumption that more difficult resources should have a different curve of acquisition. We find that foraging skills increase with age, but with diminishing marginal returns. Foraging return rates differ between resources, with difficult-to-extract resources, such as tubers and game, showing an increase in production later in life than easy-to-acquire products, such as fruits. We conclude by indicating future directions for research aiming at understanding the evolution of childhood by looking at foraging return data.  


%complete the abstract with results and conclusions




