\section{Foraging for children research: a review}

\subsection{First Studies: San and Hadza}
Despite descriptive accounts of children's foraging in many earlier ethnographic reports, the first quantitative  studies on this topic were conducted only in the late 1980s and 1990s by Blurtion Jones, Hawkes and colleagues (\cite{blurton_jones_modelling_1989, blurton_jones_differences_1994, blurton_jones_foraging_1994, blurton_jones_why_1997, hawkes_hadza_1995}).
Comparing the foraging returns of Kalahari San and Tanzanian Hadza, these studies examined how ecology shaped children’s foraging participation. Specifically, San children collected almost none of their own food resources, while Hadza children actively contributed to their own subsistence. Despite some similarities in the environment, differences in water availability, risk of predation and the proximity of resources to settlements appear to have affected how San and Hadza children and their parents coordinated their labour to maximize foraging returns (\cite{hawkes_hadza_1995}). For example, San mother-child dyads produced more kcal/hour  when children remained in camp to process \textit{mongongo} nuts than if they helped their mothers collect them. Hadza children, on the other hand, routinely accompanied their mothers to predictable berry patches, which maximized mother-child total yield. These early studies were the first to show that children’s foraging should be considered in relation to their ecological and family context, and that children could make substantial contributions to foraging returns, even if they would not always do so. To our knowledge, however, these were the only series of studies to systematically compare children’s foraging returns across cultures. 

\subsection{Optimal Foraging}
That children and adults cooperate to maximize foraging returns is one aspect of Optimal Foraging Theory (OFT). OFT investigates how animals behave when searching for food, under the premise that foraging efficiency translates to biological fitness, and thus, has been maximized by natural selection (\cite{pyke_optimal_1977}). Pioneered by \cite{macarthur_optimal_1966}, modeling the optimal use of patchy environments by an ideal predator, OFT has been successfully applied to foraging data in human populations  (see \cite{smith_anthropological_1983, hawkes_optimal_1992, winterhalder_hunter-gatherer_1981}). These studies have shed light on important aspects of human behaviour, including how much time foragers should spend in a patch of resources before moving to the next one in order to maximize returns (Patch Choice models), or where camps should be located to facilitate access to scattered resources (Central Foraging models). 
Particularly important in the human literature has been the Prey Choice model (also known as Diet Breath model), which predicts the foods that should be included in the diet given the cost of pursuit over the benefits derived from consumption. 
%\textsc{
These studies measured time and/or energy invested in finding, capturing and processing each type of animal or plant food, and their calories or nutrient composition. If a certain resource gives a return per unit time that is higher than the average return of all the other resources in the environment, it will be pursued, otherwise it will be ignored in favor of some better prey.
%} 
%\textit{These studies demonstrated that foragers coordinate their labour…..men and women face difference trade-offs and thus target difference resources… Flesh this out in 1-2 sentences to set up the next para}

Several authors tested OFT expectations on human data, focusing largely on adults (\cite{hill_foraging_1987} ). Much less research has focused on how children optimize their foraging returns. Bliege \cite{bird_children_1995, bird_ethnoarchaeology_2000, bird_constraints_2002, bird_children_2002, bird_mardu_2005}, working with Australian Meriam and Mardu people, observed that foraging children target preys matched to their size and strength. For example, Mardu children’s walking speed restricted their hunting success on sandhills. As a result, children focused their hunting efforts on lizards in rocky outcrops, where they encounter prey more quickly and with less walking. 
Subsequent research by \cite{tucker_growing_2005} among Malagasy Mikea showed that children focused on collecting younger and shallower tubers in patches abandoned by adults. Unlike adults, Mikea children did not strive for efficiency: Tucker observed a ‘food fight’ which destroyed several hours work of Mikea children’s tuber digging. Finally, \cite{crittenden_juvenile_2013} further  demonstrated that by focusing on fruit, shallow tubers, small mammals, and birds, some Hadza children could produce over 100\% of their daily caloric needs. 

These studies suggest that children’s foraging decisions reflect different trade-offs than those of adults. However, little research has examined how the resources children target change as they gain skill and strength with age.

\subsection{Cooperative Breeding}
Even among the Hadza, arguably the most active child foragers, children do not regularly forage enough to feed themselves. Instead, Hadza children, like all human juveniles, rely on the support of a wide array of alloparents, who provide care and food (\cite{crittenden_allomaternal_2008}). This is because humans are cooperative breeders, i.e. group members other than the parents help care for and provision offspring (\cite{hrdy_evolutionary_2006}). The presence of multiple caretakers is considered to have had important consequences on the evolution of life history in our species. 
For mothers, offsetting at least some of the cost of childcare has resulted in shorter inter-birth intervals than other great apes (\cite{ meehan_cooperative_2013}). For offspring, not being completely responsible for self provisioning after weaning allows more flexibility in allocation of both time and resources.
%For grandmothers…For children, having access to multiple caretakers may have resulted in the lengthening of childhood, because…  \textit{ This sentence isn’t clear enough—can you rewrite it to get at the evo logic behind alloparents=longer childhood?
%“Moreover, once the presence of a dependant offspring is not preventing further investment in fitness, the same offspring can remain much further in the dependencies of caretakers -allowing space for the evolution of childhood- without hindering too much their parent’s reproduction .”}
Research on alloparents has primarily focused on provisioning from fathers and grandmothers (\cite{hawkes_hadza_1997, gibson_helpful_2005, sear_effects_2002}). Much less is known regarding how children themselves help offset their own, and their siblings’, cost of care (\cite{reiches_pooled_2009, kramer_early_2009, kramer_pooled_2010}). And yet, their contributions may be substantial (\cite{cain_economic_1977}).
%in as  much as the activities of individual household members are coordinated  parts of a single household enterprise\textit{ Need specifics—what does Cain FIND?}. 
For example, \cite{stieglitz_household_2013} showed that Bolivian Tsimane children’s labour can substitute that of absent fathers. Among Mexican Maya, children’s contributions to the household economy through domestic chores and childcare funds “between 82 per cent and 93 per cent of their own costs and between 35 and 52 per cent of the family’s total consumption during the mid-family cycle” (\cite{kramer_variation_2002, kramer_maya_2005, kramer_childrens_2005, kramer_does_2009}). These accounts, however, mostly focus on children’s time allocation to work activities and thus, do not provide detailed accounts of direct nutritional provisioning from juveniles.






%For example, the appearance of a post reproductive period in life seems to have been selected because it allows older individuals, especially women, to care for their grandchildren and hence increase their fitness (\cite{hawkes_2005_human}). 
%Most importantly, lifting at least part of the weight of childcare from the shoulders of mothers, gives them the possibility to have a new pregnancy earlier, shortening the average inter-birth interval. Moreover, once the presence of a dependant offspring is not preventing further investment in fitness, the same offspring can remain much further in the dependencies of caretakers -allowing space for the evolution of childhood- without hindering too much their parent's reproduction. This creates the peculiarly human case of families with multiple dependent offspring as well as variably related caretakers, all of which share resources in some sort of pooled energy budget (\cite{reiches_pooled_2009, kramer_early_2009, kramer_pooled_2010}). It is well clear that the different members contributing to this shared pool might have different stakes at play, and much research has been dedicated to the different engagement of mothers, fathers as well as other stakeholders such as grandmothers. Much less is understood of how children themselves are implicated in this exchange, and their role as producers is often overlooked. As relevant examples, \cite{cain_economic_1977} points out the importance of children's contribution to the household's workload, or \cite{stieglitz_household_2013} shows that children are more likely to engage in tasks related to food in the absence of their fathers. In general, it has been demonstrated that children’s contributions to the household economy through domestic chores and childcare offsets a significant portion of their own, and their siblings’ cost of care, for example in the work among the Maya of \cite{kramer_variation_2002, kramer_maya_2005, kramer_childrens_2005, kramer_does_2009}. These accounts, however, mostly refer to time allocation data and do not provide detailed accounts of the proportion of food contributed. 

\subsection{Embodied Capital Theory}
Another aspect of human life history invoked in studies of children’s foraging is the length of the pre-reproductive period itself. The Embodied Capital Theory (ECT) proposed by \cite{kaplan_theory_2000} has focused on understanding the co-evolution of human longevity, large brains, and our skill-intensive foraging niche. Considering these factors together, proponents of ECT argue that childhood evolved as a period for learning, during which children acquire skills and knowledge  that will allow them, as adults, to successfully extract nutrient-dense resources from their environment (\cite{kaplan_theory_2000, kaplan_embodied_2001, kaplan_embodied_2003, kaplan_neural_2003, kaplan_evolution_2007, kaplan_theory_1996, kaplan_evolution_1997, kaplan_life_2006, kaplan_emergence_2002}). Research testing this hypothesis tends to focus on how experience correlated with foraging returns, independently of strength or size. According to ECT, return rates from foraging and daily amount of food collected should increase after foragers acquire the skills and knowledge to extract these resources. 
%("Our view is that human childhood is elongated by including a period of very slow physical growth, during which the brain is growing, learning is rapid, and little work is done. This is followed by adolescence, during which growth is accelerated so that the brain and body can function together in the food quest. Early adulthood is a time for vigorous work during which resource acquisition rates increase through on-the-job training.")

Support for ECT largely comes from research on adolescents and adult males engaging in a hard task such as hunting. Ache, Tsimane, and Gidra men’s hunting efficiency peaked in mid-adulthood, some ten years after peak physical strength, suggesting that experience plays a central role in the development of hunting skill (\cite{walker_age-dependency_2002, gurven_how_2006, ohtsuka_hunting_1989}). \cite{koster_life_2020} conducted the first cross-cultural study of men’s hunting returns, with data from over 1,800 individuals from over 40 societies. Their findings largely support ECT, with age of peak hunting success averaging between 30 and 25 years of age. However, the authors note considerable individual and cross-cultural variation in hunting returns, likely due to environment, prey type, prey encounters, and individual motivation. 

Studies focusing on all foraging returns rather than solely hunting, and on younger children rather than adolescents and adults, have provided more mixed support for ECT.

\cite{bock_learning_2002} found that competence, acquired through both experience and physical growth, is reached earlier for simpler tasks compared to more complex ones, suggesting support for ECT. Other studies find limited evidence for the importance of skill acquisition of children’s foraging success. 
For example, \cite{ bird_constraints_2002, bird_children_2002, bird_mardu_2005} argue that Mardu and Meriam children’s foraging returns are primarily restricted by their small size.
\cite{blurton_jones_selection_2002} further found that Hadza children who had attended boarding schools, and thus, had little bush experience, did not underperform in tubber-digging compared to bush-residing counterparts. Finally, while \cite{crittenden_juvenile_2013} found that Hadza foraging returns increased with age, young children can collect enough food to exceed their daily caloric needs if needed, suggesting that motivation rather than age influences children’s returns. Taken together, these studies point to the effects of growth and skill acquisition on children’s foraging success will depend on the types of resources available in their ecological niche, which requires cross-cultural investigation. 
%\textit{I think we need a stronger closing here…}

\subsection{The Present Study}
In summary, while there is a long history of studying children’s foraging returns, ecologically-grounded and empirical cross-cultural work on children’s direct nutritional provisioning is still needed to answer outstanding questions regarding children’s optimal foraging, alloparenting, and skill acquisition. In their study of hunting returns, \cite{koster_life_2020} demonstrated the strength of a cross-cultural comparative approach to understanding the development of human foraging, and its implications for resolving current human life history debates. Here, we expand beyond hunting by presenting the first meta-analysis of child and adolescent foraging returns. Our study has two main goals. First, we examine age-dependent variation in children’s foraging returns, in order to elucidate the factors that may constrain and enable children’s foraging participation across cultures. Second, we assess how methodological and analytical differences introduce heterogeneity in studies of children’s foraging returns, in order to promote the comparability of future studies.  
